\documentclass{article}

\usepackage[T2A]{fontenc}
\usepackage[utf8]{inputenc}
\usepackage[russian]{babel}
\parindent 0pt
\parskip 8pt
\usepackage{setspace}
\usepackage{etaremune}
\usepackage{amsmath}
\usepackage{amssymb}
\usepackage{amsfonts}
\usepackage[left=2.3cm, right=2.3cm, top=2.7cm, bottom=2.7cm, bindingoffset=0cm]{geometry}
\usepackage{latexsym}
\usepackage[unicode, pdftex]{hyperref}
\usepackage{xcolor}
\usepackage{graphicx}
\usepackage{mathtools}
\graphicspath{ {./images/} }

\doublespacing

\everymath{\displaystyle}

\begin{document}

\newcommand{\RM}[0]{\mathbb{R}^m}
\newcommand{\dist}[0]{\mathrm{dist} $\ $}
\newcommand{\rang}[0]{\mathrm{rang} $\ $}
\newcommand{\grad}[0]{\mathrm{grad} $\ $}
\newcommand{\Lin}[0]{\mathrm{Lin} $\ $}

\tableofcontents

\newpage 

\part{Определения}

    \newpage

    \section{Диффеоморфизм}
    
        \textit{Область} в $\RM$ --- открытое связное множество.
        
        Пусть $f : O \subset \RM \rightarrow \RM$~--- \textit{диффеоморфизм}, где $O$ --- область, если:
        
        \begin{enumerate}
        
            \item $f$~--- обратима;
            
            \item $f$~--- дифференцируема;
            
            \item $(f^{-1})$~--- тоже дифференцируема.
            
        \end{enumerate}
        
    \newpage
    
    \section{Формулировка теоремы о локальной обратимости в терминах систем уравнений}
    
    $
            \begin{cases}
            
                f_1(x_1, \ldots, x_m) = y_1 \\
                
                \ldots \\
                
                f_m(x_1, \ldots, x_m) = y_m
            
            \end{cases}
    $
        
        Все $f_i$~--- гладкие.
        
        Пусть при $y = (b_1, \ldots, b_m)$ существует единственное решение $x = (a_1, \ldots, a_m)$, что $\det \left( \frac{\partial f_i}{\partial x_j} (a) \right) \neq 0$.
        
        Тогда для $y_0$ близких к $(b_1, \ldots, b_m)$ существует решение $(x_1, \ldots, x_m)$ близкое к $(a_1, \ldots, a_m)$ и зависимое от $y$, причём оно гладкое.
        
    \newpage
    
    \section{Формулировка теоремы о неявном отображении в терминах систем уравнений}
    
    $
            \begin{cases}
            
                f_1(x_1, \ldots, x_m, y_1, \ldots, y_n) = 0 \\
                
                \ldots \\
                
                f_n(x_1, \ldots, x_m, y_1, \ldots, y_n) = 0
                
            \end{cases}
    $
        
        $x = a$ и $y = b$ удовлетворяют системе уравнений, а также $f_i$~--- функции класса $C^r$, также
        
        $\det \left( \dfrac{\partial f_i}{\partial y_j} (a, b) \right) \neq 0$. Тогда
        
        $\exists U(a)$ и $V(b)$ такое, что $\exists ! \varphi: U(a) \rightarrow V(b)$ класса $C^r$, что $\forall x \in U(a)$ верно $(x, \varphi(x))$~--- решение этой системы.
        
    \newpage
    
    \section{Простое $k$-мерное гладкое многообразие в $\RM$}
            
        \begin{enumerate}
        
            \item $M \subset \RM$~--- простое $k$-мерное многообразие в $\RM$ (непрерывное), если оно гомеоморфно открытому множеству из $\mathbb{R}^k$, т.е.:
            
                $\exists O \subset \mathbb{R}^k$ и $\exists \Phi : O \rightarrow M$ такое, что
            
                \begin{itemize}
            
                    \item $\Phi$~--- сюрьекция;
                
                    \item $\Phi$~--- непрерывное;
                
                    \item $\Phi$~--- обратимо и $\Phi^{-1}$~--- непрерывно.
                
                \end{itemize}
            
            \item $M \subset \RM$~--- простое, $k$~---мерное, $C^r$~--- гладкое многообразие, если:
            
                $\exists O \subset \mathbb{R}^k$, $\Phi : O \rightarrow \RM$:
                
                \begin{itemize}
                
                    \item $\Phi$~--- гомеоморфизм;
                    
                    \item $\Phi \in C^r(O, \RM)$~--- это гладкость;
                    
                    \item $\forall t \in O$ верно, что $\rang \Phi'(t) = k$.
                    
                \end{itemize}
        
        \end{enumerate}
    
    \newpage
    
    \section{Касательное пространство к $k$-мерному многообразию в $\RM$}
    
        $\Phi : O \subset \mathbb{R}^k \rightarrow \RM$~--- $C^r$-параметризация $U(p) \cap M$, $p \in M$, $\Phi(t_0) = p$. Тогда $\Phi'(t_0) \left( \mathbb{R}^k \right)$ называется касательным пространством к $k$-мерному многообразию $M$ в точке $p$. Обозначается $Tp(M) = \left\{ \Phi'(t_0) h, h \in \mathbb{R}^k \right\}$.
        
    \newpage
    
    \section{Относительный локальный максимум, минимум, экстремум}
    
        $f : E \subset \mathbb{R}^{m + n} \rightarrow \mathbb{R}$,
        
        $\Phi : E \rightarrow \mathbb{R}^n$, тогда
        
        $x_0 \in E$, $\Phi(x_0) = 0$~--- точка относительного локального максимума $f$, если $\exists U (x_0) : \forall x \in U(x_0) \cap E, \Phi(x) = 0$ $f(x) \leq f(x_0)$. 
        
        Аналогично определяется минимум.
        
    \newpage
    
    \section{Формулировка достаточного условия относительного экстремума}
    
        $f : E \subset \mathbb{R}^{m + n} \rightarrow \mathbb{R}$, $\Phi : E \rightarrow \mathbb{R}^n$, $f$, $\Phi \in C^1$, 
        
        $a \in E : \rang \Phi'(a) = n$,
        
        $\lambda \in \mathbb{R}^n$ и верно
        
        $
            \begin{cases}
    
                f'(a) - \lambda \Phi'(a) = 0 \\
                
                \Phi(a) = 0
                
            \end{cases}
        $
        
        Если $h = ( h_x, h_y ) \in \mathbb{R}^{m + n}$ удовлетворяет $\Phi'(a) h = 0$, то можно выразить $h_y = \psi(h_x)$.
        
        Рассмотрим квадратичную форму $Q(h_x) = d^2 G(a, \left( h_x, \psi (h_x ) \right) )$, где $G = f - \lambda \Phi$ (форма Лагранжа).
        
        $Q$~--- положительно определенная, значит точка локального минимума, если отрицательно определенная~--- точка локального максимума, $Q$~--- неопределенная~--- нет экстремума, в остальных случаях непонятно.
        
    \newpage
    
    \section{Поточечная сходимость последовательности функций на множестве }
    
        $f_n : X \rightarrow \mathbb{R}$, если $\exists f : E \subset X \rightarrow \mathbb{R}$, что для любого $x_0 \in E$ предел $\lim\limits_{n \rightarrow +\infty} f_n(x_0) = f(x_0)$, то
        
        $f_n \xrightarrow[n \rightarrow +\infty]{} f$ сходится поточечно на $E$.
        
    \newpage
    
    \section{Равномерная сходимость последовательности функций на множестве }
    
        $f$, $f_n : X \rightarrow \mathbb{R}$, $E \subset X$, тогда $f_n$~--- равномерно сходится на $E$ к функции $f$ если
        
        $M_n := \sup\limits_{x \in E} \left| f_n(x) - f(x) \right| \xrightarrow[n \rightarrow +\infty]{} 0$.
        
        Обозначается как $f_n \rightrightarrows f$ на множестве $E$.
        
    \newpage
    
    \section{Равномерная сходимость функционального ряда}
    
        \begin{enumerate}
        
            \item Функциональный ряд сходится поточечно на $E$, если для любого $x \in E$ сумма $\sum\limits^{+\infty}_{n = 0} u_n(x)$~--- сходится к сумме $S(x)$;
            
            \item Функциональный ряд сходится равномерно на $E$ (к сумме $S(x)$), если $S_n(x) \rightrightarrows S(x)$ на $E$, где $S_n(x)$~--- последовательность частичных сумм.
            
        \end{enumerate}
        
    \newpage
    
    \section{Формулировка критерия Больцано--Коши для равномерной сходимости}
    
        Функциональный ряд $\sum u_n$ равномерно сходится на $E$ эквивалентно следующему утверждению:
        
        $$\forall \varepsilon > 0 : \exists N : \forall m \geq n \geq N \ \textrm{и} \ \forall x \in E : \left| \sum\limits^m_{k = n} u_k(x) \right| < \varepsilon$$
        
    \newpage
    
    \section{Степенной ряд, радиус сходимости степенного ряда, формула Адамара}
    
        $a_n \in \mathbb{R}$ или $\mathbb{C}$, $B(z_0, r) \subset \mathbb{R}$ или $\mathbb{C}$, тогда $\sum\limits^{+\infty}_{n = 0} a_n (z - z_0)^n$ называют \textbf{степенным рядом}.
        
        Назовём $R$ \textbf{радиусом сходимости} степенного ряда, если:
        
            \begin{itemize}
            
                \item при $|z - z_0| < R$ ряд абсолютно сходится;
                
                \item при $|z - z_0| > R$ ряд расходится.
                
            \end{itemize}
        
        $R = \dfrac{1}{\overline{\lim\limits_{n \rightarrow +\infty}} \sqrt[n]{|a_n|}}$~--- формула Адамара.
    
    \newpage
    
    \section{Кусочно-гладкий путь}
    
        $\gamma : [a, b] \rightarrow \RM$~--- кусочно-гладкий путь, если существует такое дробление $a = t_0 < t_1 < \ldots < t_n = b$, что для любого $i \in [1, n]$ путь $\gamma \big|_{[t_{i - 1}, t_i]}$~--- гладкий (в точка $t_{i - 1}$ и $t_i$ есть односторонние производные).
        
    \newpage
    
    \section{Векторное поле}
    
        $V : E \subset \RM \rightarrow \RM$~--- векторное поле. По умолчанию считается, что $V$~--- непрерывное.
        
    \newpage
    
    \section{Интеграл векторного поля по кусочно-гладкому пути}
    
        $V$~--- векторное поле в $E$, $E$~--- открытое, $\gamma$~--- кусочно-гладкий путь в $E$. Тогда
        
        $$I(V, \gamma) := \int\limits^b_a \langle V \left( \gamma(t) \right), \gamma' (t) \rangle dt = \int\limits^b_a V_1 \left( \gamma(t) \right) \gamma'_1(t) + \ldots + V_m \left( \gamma(t) \right) \gamma'_m(t) dt$$
        
        интеграл векторного поля $V$ по кусочно-гладкому пути $\gamma$.
        
    \newpage
    
    \section{Потенциал, потенциальное векторное поле}
    
        $O \subset \RM$~--- область, $V : O \rightarrow \RM$. Тогда $V$~--- потенциальное векторное поле, а $f$~--- его потенциал, если $f \in C^1 \left( O, \mathbb{R} \right)$ и $\grad f = V$ в области $O$.
        
    \newpage
    
    \section{Локально потенциальное векторное поле}
    
        $V : O \subset \RM \rightarrow \RM$~--- локально потенциальное векторное поле, если $O$~--- область, а $\forall x \in O : \exists U(x)$, что $V$ в $U(x)$~--- потенциальное поле.
        
    \newpage
    
    \section{Похожие пути}
    
        $V$~--- локально потенциальное векторное поле, $\gamma$, $\overline{\gamma} : [a, b] \rightarrow O$~--- непрерывны. Тогда $\gamma$ и $\overline{\gamma}$~--- похожи, если у этих путей имеется одинаковая $V$-гусеница, т.е. существует такие дробления $t_0$, $t_1$, $\ldots$, $t_n$ и $\overline{t_0}$, $\overline{t_1}$, $\ldots$, $\overline{t_n}$, что $a = t_0 = \overline{t_0}$ и $b = t_n = \overline{t_n}$, и шары $B_1$, $B_2$, $\ldots$, $B_k$, а также $\gamma \big|_{[t_{k - 1}, t_k]} \subset B_k$ и $\overline{\gamma} \big|_{[\overline{t_{k - 1}}, \overline{t_k}]} \subset B_k$.
        
    \newpage
    
    \section{Интеграл локально-потенциального векторного поля по произвольному пути}
    
        $I(V, \gamma) = I(V, \overline{\gamma}) = \int\limits^b_a \langle V \left( \overline{\gamma}(t) \right), \overline{\gamma}'(t) \rangle dt$~--- интеграл локально векторное поля $V$ по произвольному пути $\gamma$, где $\overline{\gamma}$~--- похожий на $\gamma$ кусочно-гладкий путь. В условиях соответствующей леммы такой всегда существует.
        
    \newpage
    
    \section{Гомотопия путей связанная и петельная}
    
        $\gamma_0$, $\gamma_1 : [a, b] \rightarrow O$, тогда гомотопия~--- это отображение $\Gamma : [a, b] \times [0, 1] \rightarrow O$~--- непрерывное, такое что $\Gamma(\cdot, 0) = \gamma_0(\cdot)$ и $\Gamma(\cdot, 1) = \gamma_1(\cdot)$.
        
        \textit{Гомотопия связанная}: $\gamma_0(a) = \gamma_1(a)$, $\gamma_0(b) = \gamma_1(b)$ и $\forall u \in [0, 1]$ $\Gamma(a, u) = \gamma_0(a)$ и $\Gamma(b, u) = \gamma_0(b)$.
        
        \textit{Гомотопия петельная}: $\gamma_0(a) = \gamma_0(b)$, $\gamma_1(a) = \gamma_1(b)$ и $\forall u \in [0, 1]$ $\Gamma(a, u) = \Gamma(b, u)$.
        
    \newpage
    
    \section{Односвязная область}
    
        $O \subset \RM$~--- односвязная область, если $O$~--- область и любой замкнутый путь гомотопен постоянному.
        
    \newpage
    
    \section{Полукольцо, алгебра, сигма-алгебра}
    
        $X$~--- множество, $\mathcal{P} \subset 2^X$~--- \textit{полукольцо} если:
        
        \begin{enumerate}
        
            \item $\varnothing \in \mathcal{P}$;
            
            \item $\forall A$, $B \in \mathcal{P} \Rightarrow A \cap B \in \mathcal{P}$;
            
            \item $\forall A_1$, $A_2 \in \mathcal{P}$ существует конечное число $B_1$, $B_2$, $\ldots$, $B_k \in \mathcal{P}$, что $A_1 \setminus A_2 =  \bigsqcup\limits^k_{i = 1} B_i$.
            
        \end{enumerate}
    
        $\mathcal{A}$~--- \textit{алгебра} подмножеств $X$, если:
        
        \begin{enumerate}
        
            \item $A$, $B \in \mathcal{A} \Rightarrow A \setminus B \in \mathcal{A}$;
            
            \item $X \in \mathcal{A}$.
            
        \end{enumerate}
        
        $\sigma$-\textit{алгебра} $\mathcal{A}$, если это \textit{алгебра} и ещё выполнено третье свойство:
        
            $\forall A_1$, $A_2$, $A_3$, $\ldots \in \mathcal{A} \Rightarrow \bigcup\limits_{i = 1}^{+\infty} A_i \in \mathcal{A}$.
            
        \textit{Можно вместо объединения потребовать пересечение, поскольку из одного следует другое}
        
    \newpage
    
    \section{Объем}
    
        $\mu : \mathcal{P} \rightarrow \overline{\mathbb{R}}$~--- \textit{аддитивная}, если:
        
        \begin{enumerate}
        
            \item $\mu$ не принимает одновременно бесконечности разных знаков;
            
            \item $\mu(\varnothing) = 0$;
            
            \item $\forall A_1$, $A_2$, $\ldots$, $A_n \in \mathcal{P}$~--- дизъюнкты, если $A = \bigsqcup A_n \in \mathcal{P}$, тогда $\mu A = \mu A_1 + \mu A_2 + \ldots + \mu A_n$.
        
        \end{enumerate}
        
        $\mu$~--- \textit{объём}, если:
        
        \begin{enumerate}
        
            \item $\mu : \mathcal{P} \rightarrow \overline{\mathbb{R}}$;
            
            \item $\mu \geq 0$;
            
            \item $\mu$~--- аддитивная.
            
        \end{enumerate}
        
    \newpage
    
    \section{Ячейка}
    
        $a, b \in \RM$, $[a, b) = \left\{ x \in \RM : \forall i : a_i \leq x_i < b_i \right\}$~--- ячейка.
        
    \newpage
    
    \section{Классический объем в $\RM$}
    
        $\mu [a, b) = \prod\limits^m_{i = 1} (b_i - a_i)$.
        
    \newpage
    
    \section{Мера, пространство с мерой}
            
        $\mu : \mathcal{P} \rightarrow \overline{\mathbb{R}}$~--- мера, если $\mu$~--- \textit{объём}, а также выполнено свойство \textit{счётной аддитивности}, т.е.:
        
        $\forall A$, $A_1$, $A_2$, $\ldots$, где $A_i$~--- дизъюнкты и $A = \bigsqcup\limits^{+\infty}_{i = 1} A_i \Rightarrow \mu A = \sum\limits^{+\infty}_{i = 1} \mu A_i$.
        
        $(X, \mathcal{A}, \mu)$~--- пространство с мерой, если $\mathcal{A}$~--- $\sigma$-алгебра на множестве $X$, а $\mu$~--- мера на $\mathcal{A}$.
        
    \newpage
    
    \section{Полная мера}
    
        $(X, \mathcal{A}, \mu)$, $\mu$~--- полная мера, если $\forall E \in \mathcal{A}$ и $\mu E = 0$ верно, что $\forall e \in E : e \in \mathcal{A}$ и $\mu e = 0$.
        
    \newpage
    
    \section{Сигма-конечная мера}
    
        $(X, \mathcal{P}, \mu)$, $\mu$~--- $\sigma$-конечная, если можно представить $X = \bigcup\limits^{+\infty}_{k = 1} B_k$, где $\mu B_k < +\infty$.
        
    \newpage
    
    \section{Дискретная мера}
    
        $X$~--- множество, $A_1$, $A_2$, $\ldots$~--- точки множества $X$, $h_1$, $h_2$, $\ldots \geq 0$, $\mathcal{P} = 2^X$, и для любого $B \subset \mathcal{P}$ $\mu B = \sum\limits_{i : A_i \in B} h_i$.
        
\newpage

\part{Теоремы}

    \newpage
    
    \section{Лемма о ''почти локальной инъективности''}
    
        $F : O \subset \RM \rightarrow \RM$ --- дифференцируема в точке $x_0 \in O$, 
        
        $\det F'(x_0) \neq 0$,
        
        $O$ --- область.
        
        Тогда $\exists c, \delta > 0 : \forall h : \left| h \right| < \delta : \left| F(x_0 + h) - F(x_0) \right| \geq c \cdot \left| h \right|$
        
        \subsection{Доказательство}
        
            $\left| F(x_0 + h) - F(x_0) \right| = \left| F'(x_0)h + \alpha(h) \left| h \right| \right| \geq \left| F'(x_0) h \right| - \left| \alpha(h) \right| \left| h \right| \geq (\widetilde{c} - \left| \alpha(h) \right|) \left| h \right| \geq \frac{c}{2} \left| h \right|$
        
            Возьмём в качестве $\widetilde{c} = \dfrac{1}{\| (F'(x_0))^{-1} \|}$.
        
            Пусть при $|h| < \delta$ будет верно, что $\left| \alpha(h) \right| < \frac{\widetilde{c}}{2}$.
        
    \newpage
    
    \section{Теорема о сохранении области}
    
        $F : O \subset \RM \rightarrow \RM$, где $O$~--- открыто,
        
        для любого $x \in O$ выполняется $\det F'(x) \neq 0$. Тогда $F(O)$~--- открыто.
        
        \subsection{Доказательство}
        
            Пусть $x_0 \in O$ и $y_0 = F(x_0) \in F(O)$, необходимо проверить, что $y_0$~--- внутренняя точка $F(O)$.
            
            По лемме о ''почти локальной инъективности'' существуют такие $c$ и $\delta$, что для любого $h \in \overline{B(0, \delta)}$ верно $\left| F(x_0 + h) - F(x_0) \right| \geq c | h |$ (и в частности $F(x_0 + h) \neq F(x_0)$ при $| h | = \delta$).
            
            $r := \frac{1}{2} \dist \left( y_0, F(S(x_0, \delta)) \right) > 0$
            
            Проверим, что $B(y_0, r) \subset F(O)$. Пусть $y \in B(y_0, r)$ и $g(x) := \left| F(x) - y \right|$~--- функция на $\overline{B(x_0, \delta)}$.
            
            \begin{enumerate}
            
                \item На $S(x_0, \delta)$ верно, что $\left| F(x) - y \right| \geq r$
                
                \item При $x = x_0$ выполняется, что $\left| F(x_0) - y \right| = | y_0 - y | < r$, по теореме Вейерштрасса $g$ достигается минимума внутри шара $B(x_0, \delta)$.
                
            \end{enumerate}
            
            Пусть $l : x \mapsto \left| F(x) - y \right|^2$~--- достигает минимума таким же образом.
            
            Найдём минимум с помощью необходимого условию экстремума, т.е. производная должна быть равна $0$.
            
            \begin{equation*}
                \begin{cases}
   
                    l'_{x_1} = 0 & 2 (f_1 (x_1, \ldots, x_m) - y_1) \cdot \frac{\partial f_1}{\partial x_1} + \ldots + 2 (f_m (x_1, \ldots, x_m) - y_m) \cdot \frac{\partial f_m}{\partial x_1} = 0 \\
                    
                    \ldots \\
                    
                    l'_{x_m} = 0 & 2 (f_1 (x_1, \ldots, x_m) - y_1) \cdot \frac{\partial f_1}{\partial x_m} + \ldots + 2 (f_m (x_1, \ldots, x_m) - y_m) \cdot \dfrac{\partial f_m}{\partial x_m} = 0
                
                \end{cases}
            \end{equation*}
            
            Поскольку матрица $F'(x)$ невырожденная по условию, то получаем, что $f_i(x) - y = 0$ для всех $i$.
            
    \newpage
    
    \section{Следствие о сохранении области для отображений в пространство меньшей размерности}
    
        $F : O \subset \RM \rightarrow \mathbb{R}^l$, где $l < m$ и $F \in C^1(O)$,
        
        $\rang F'(x) = l$ при всех $x \in O$.
        
        Тогда $F(O)$~--- открыто.
        
        \subsection{Доказательство}
        
            В точке $x_0$ и в окрестности ранг реализован на первых $l$ столбцах. 
            
            Пусть $\widetilde{F} = \begin{pmatrix} F \\ x_{l + 1} \\ \ldots \\ x_{m} \end{pmatrix} : O \rightarrow \RM$
            
            $(x_1, \ldots, x_m) \mapsto \left( F(x_1, \ldots, x_m), x_{l + 1}, \ldots, x_m \right)$
            
            $\det \widetilde{F}(x_0) = \det F(x_0) \neq 0$, а также $\forall x \in U(x_0)$. 
            
            Значит $\widetilde{F}(U(x_0))$ открыто в $\RM$, а $F(U(x_0))$~--- проекция на $\mathbb{R}^l$.
            
    \newpage
    
    \section{Теорема о гладкости обратного отображения}
    
        $T \in C^r \left( O, \RM \right)$ ($r = 1$, $2$, $\ldots$, $+\infty$).
        
        Пусть $T$~--- обратимо, $\det T'(x) \neq 0$ при $x \in O$. Тогда
        
        $T^{-1} \in C^r$ и при этом $\left( T^{-1} \right)'(y_0) = \left( T'(x_0) \right)^{-1}$ при $y_0 = T(x_0)$.
        
        \subsection{Доказательство}
        
            Индукция по $r$:
            
            \begin{itemize}
            
                \item База $r = 1$:
                
                    $S = T^{-1}$~--- обратное отображение, $S$~--- непрерывно (по теореме о сохранении области).
                    
                    $O$~--- открытое $\Rightarrow$ $T(O)$~--- открытое, значит $T: \RM_{(1)} \rightarrow \RM_{(2)}$, а $S : \RM_{(2)} \rightarrow \RM_{(1)}$, значит и $S^{-1}$~--- тоже открытое.
                    
                    $T(O) = O_1$, $y_0 \in O_1$, верно ли, что $S$~--- дифференцируема в $y_0$? Обозначим $A = T'(x_0)$.
                    
                    По лемме о почти локальной инъективности $\exists c, \delta : x \in B(x_0, \delta)$, что $\left| T(x) - T(x_0) \right| \geq c | x - x_0 |$
                    
                    По определению дифференцирования $T(x) - T(x_0) = A(x - x_0) + \alpha(x) | x - x_0 |$
                    
                    $S(y) - S(y_0) = A^{-1} (y - y_0) + A^{-1} \alpha \left( S(y) \right) \left| S(y) - S(y_0) \right|$.
                    
                    Пусть $\beta = A^{-1} \alpha \left( S(y) \right) \left| S(y) - S(y_0) \right|$
                    
                    Пусть $y$ близко к $y_0$ : $| x - x_0 | = \left| S(y) - S(y_0) \right| < \delta$~--- по непрерывности $S'$.
                    
                    $| \beta(y) | = |x - x_0| \cdot \left| A^{-1} \alpha \left( S(y) \right) \right| \leq \frac{1}{c} \left| T(x) - T(x_0) \right| \cdot \| A^{-1} \| \alpha \left( S(y) \right) | = \dfrac{\| A^{-1} \|}{c} \left| \alpha \left( S(y) \right) \right| |y - y_0| = o(|y - y_0|)$ при $y \rightarrow y_0$.
                    
                    $\left| T(x) - T(x_0) \right| \geq c | x - x_0 | \Rightarrow | x - x_0 | \leq \frac{1}{c} \left| T(x) - T(x_0) \right|$.
                    
                    $S' : y \xrightarrow{C^1} T^{-1} (y) = x \xrightarrow{C^1} T'(x) \xrightarrow{C^{\infty}} \left(T'(x)\right)^{-1} = S'$.
                    
                \item Индукционный переход без доказательства:
                
                    $r = 1 \Rightarrow r = 2$, т.е. $T \in C^2 \Rightarrow S \in C^2$, т.е. $S' \in C^1$, а также $T \in C^1$ и $S \in C^1$.
                    
            \end{itemize}
            
    \newpage
    
    \section{Лемма о приближении отображения его линеаризацией}
    
        $T \in C^1 \left( O, \RM \right)$, $x_0 \in O$.
        
        Тогда для любого $h$ $\left| T(x_0 + h) - T(x_0) - T'(x_0) h \right| \leq M \cdot |h|$, где $M = \sup\limits_{z \in [x_0, x_0 + h]} \| T'(z) - T'(x_0) \|$.
        
        \subsection{Доказательство}
        
            $\left| F(x) - F(x_0) \right| \leq \sup\limits_{z \in [x_0, x]} \| F^{-1} (z) \| \cdot | x - x_0 |$~--- по теореме Лагранжа.
            
            $F(x) = T(x) - T'(x_0) \cdot X$
            
            $F'(x) = T'(x) - T'(x_0)$
            
            $\left| T(x_0 + h) - T(x_0) - T'(x_0) h \right| = \left| F(x_0 + h) - F(x_0) \right| \leq \sup\limits_{z \in [x_0, x_0 + h]} \| F'(z) \| |h|$.
            
    \newpage
    
    \section{Теорема о локальной обратимости}
    
        $T \in C^1 \left( O, \RM \right)$, $x_0 \in O$ и $\det T'(x_0) \neq 0$. Тогда
        
        $\exists U(x_0) : T \big|_{U(x_0)}$~--- диффеоморфизм.
        
        \subsection{Доказательство}
        
            Достаточно доказать, что $\exists U(x_0)$, что $T \big|_{U(x_0)}$~--- обратимо (и для любого $x \in U(x_0)$ $\det T'(x) \neq 0$).
            
            $T'(x_0)$~--- обратимо, значит $\exists c > 0 : \forall h$ $\left| T'(x_0) h \right| \geq c |h|$, где $c = \dfrac{1}{\| T'(x_0)^{-1} \|}$.
            
            Возьмём $U = B(x_0, r) \subset O$ так, что при $x \in U$ и было верным:
            
                $\det T'(x) \neq 0$ и $\| T'(x) - T'(x_0) \| < \frac{c}{4}$.
                
            Проверим, что $T \big|_U$~--- взаимно-однозначное отображение.
            
            $x, y \in U$ и $y = x + h$
            
            $T(y) - T(x) = \left( T(x + h) - T(x) - T'(x) h \right) + \left( T'(x)h - T'(x_0)h \right) + T'(x_0)h$
        
            (Здесь и ниже римскими цифрами отображается номер скобки в выражении сверху)
            
            $\left| T(y) - T(x) \right| \geq \left| T'(x_0) h \right| - | \mathrm{I} | - | \mathrm{II} | \geq c |h| - \frac{c}{2} |h| - \frac{c}{4} |h| = \frac{c}{4} |h| \neq 0$.
            
            $| \mathrm{I} | \geq M |h|$
            
            $\left| T(x_0 + h) - T(x_0) - T'(x_0) h     \right| \leq M |h|$
            
            $M = \sup \| T'(z) - T'(x_0) \|$, $z \in [x_0, x_0 + h]$
            
            $M \leq \frac{c}{2}$.
            
    \newpage
    
    \section{Теорема о неявном отображении}
    
        $F : O \subset \mathbb{R}^{m + n} \rightarrow \mathbb{R}^n$, $F \in C^r \left( O, \mathbb{R}^n \right)$,
        
        $(a, b) \in O$ и $F(a, b) = 0$,
        
        $\det F'_y (a, b) \neq 0$.
        
        Тогда:
        
        \begin{enumerate}
        
            \item Существует открытое $P \in \RM$, $a \in P$ и также существует открытое $Q \in \mathbb{R}^n$, $b \in Q$ такие, что
            
                $\exists ! \varphi : P \rightarrow Q$~--- $C^r$-гладкое, такое, что $\forall x \in P$ $F(x, \varphi(x)) = 0$.
                
            \item $\varphi'(x) = - \left(F'_y (x, \varphi(x)) \right)^{-1} \cdot F'_x (x, \varphi(x))$.
    
        \end{enumerate}
        
        \subsection{Доказательство}
        
            \begin{enumerate}
            
                \item $\Phi : O \rightarrow \mathbb{R}^{m + n}$
                
                    $(x, y) \mapsto \left( x, F(x, y) \right)$
                    
                    $\Phi' = \begin{pmatrix} \mathbb{E} & \mathrm{0} \\ F'_x & F'_y \end{pmatrix}$, $\det \Phi'(a, b) \neq 0$
                    
                    $\exists \widetilde{U}(a, b) : \Phi \big|_{\widetilde{U}}$~--- диффеоморфизм.
                    
                    $\widetilde{U} = P_1 \times Q$, где $a \in P_1$, $b \in Q$.
                    
                    \begin{enumerate}
                    
                        \item $\widetilde{V} = \Phi \left( \widetilde{U} \right)$~--- открыто;
                        
                        \item $\exists \psi = \Phi^{-1} : \widetilde{V} \rightarrow \widetilde{U}$;
                        
                        \item $\Phi$ не меняет первую координату, значит $\psi$ тоже не меняет,
                        
                            $\psi(u, v) = (u, H(u, v) )$, $H : \widetilde{V} \rightarrow \mathbb{R}^n$, $H \in C^r$;
                            
                        \item ''ось $x$'' и ''ось $u$'' одно и то же $\RM$,
                        
                            $P := \left( \RM \times \left\{ O_n \right\} \right) \cap \widetilde{V}$~--- открыто в $\RM$;
                            
                        \item $\psi (x) := H(x, 0) : P \rightarrow Q : F(x, \psi(x) ) = 0$~--- единственно,
                        
                            $x \in P$, $y \in Q$ $F(x, y) = 0$, $(x, y) = \psi \left( \Phi (x, Y) \right) = \psi(x, 0) = (x, H(x, 0))$.
                            
                    \end{enumerate}
                    
                \item $F(x, \varphi(x)) = 0$, $F \circ H = 0$,
                
                    $\begin{pmatrix} F'_x & F'_y \end{pmatrix} \begin{pmatrix} E \\ \varphi'(x) \end{pmatrix} = 0 \Rightarrow F'_x + F'_y \varphi'(x) = 0$,
                    
                    $F'_y \varphi' = -F'_x$
                    
                    $\varphi' = - \left( F'_y \right)^{-1} F'_x$
                
            \end{enumerate}
            
    \newpage
    
    \section{Теорема о задании гладкого многообразия системой уравнений}
            
            $M \subset \RM$, зафиксируем $1 \leq k < m$ и $1 \leq r \leq +\infty$.
            
            Тогда $\forall p \in M$ эквивалентны следующие два утверждения:
            
            \begin{enumerate}
            
                \item $\exists U \subset \RM$~--- открытое, $p \in U$,
                
                    $M \cap U$~--- простое $k$-мерное $C^r$-гладкое многообразие;
                    
                \item $\exists \widetilde{U} \subset \RM$~--- открытое, $p \in \widetilde{U}$,
                
                    что существуют функции $f_1$, $f_2$, $\ldots$, $f_{m - k} : \widetilde{U} \rightarrow \mathbb{R} \in C^r$ такие, что
                    
                    $x \in M \cap \widetilde{U} \Longleftrightarrow f_1(x) = 0$, $f_2(x) = 0$, $\ldots$, $f_{m - k}(x) = 0$ и ($\grad f_1(p)$, $\ldots$, $\grad f_{m - k}(p)$)~--- ЛНЗ.
                    
            \end{enumerate}
            
            \subsection{Доказательство}
            
                \begin{itemize}
                
                    \item $1 \Rightarrow 2$:
                    
                        Существует параметризация $\Phi \in C^r \left( O \subset \mathbb{R}^k, \RM \right)$,
                        
                        $\varphi_1, \ldots, \varphi_m$~--- координатные функции $\Phi$ и $p = \Phi(t_0)$, $\rang \Phi'(t_0) = k$.
                        
                        Можно считать, что $\left( \dfrac{\partial \varphi_i}{\partial t_j} (t_0) \right)$~--- невырождена.
                        
                        $\RM = \mathbb{R}^k \times \mathbb{R}^{m - k}$.
                        
                        $L : \RM \rightarrow \mathbb{R}^k$~--- проекция, $x \mapsto (x_1, \ldots, x_k)$.
                        
                        $L \circ \Phi$ имеет невырожденный производный оператор в точке $t_0$.
                        
                        $\exists w(t_0)$~--- окрестность $t_0$, $\exists V \in \mathbb{R}^k$~--- открытое и $L \circ \Phi : w \rightarrow V$~--- диффеоморфизм.
                        
                        $L(w) \rightarrow V$~--- взаимно-однозначеное отображение, т.е. $\Phi(w)$~--- график некоторого отображения $H : V \rightarrow \mathbb{R}^{m - k}$.
                        
                        Пусть $\psi = \left( L \circ \Phi \right)^{-1} : V \rightarrow w$, $\psi \in C^r$.
                        
                        Если $\widetilde{x} \in V$, то $\left( \widetilde{x}, H(\widetilde{x}) \right) = \Phi (w(\widetilde{x})) \Rightarrow H \in C^r$.
                        
                        $\Phi(w)$~--- открыто в $M$, $\exists$ открытое $\widetilde{U} \in \RM$ такое, что $\widetilde{U} \cap M = \Phi(w)$ (можно считать, что $\widetilde{U} \subset V \times \mathbb{R}^{m - k}$.
                        
                        $f_j : \widetilde{U} \rightarrow \mathbb{R}$, $f_j(x) = H_j \left( L(x) \right) - x_{k + j}$, если $x \in \widetilde{U} \cap M \Leftrightarrow$ все $f_j(x) = 0$.
                        
                        $\begin{pmatrix} \dfrac{\partial H_1}{\partial x_1} & \ldots & \dfrac{\partial H_1}{\partial x_k} & -1 & 0 & \ldots & 0 \\ \dfrac{\partial H_2}{\partial x_1} & \ldots & \dfrac{\partial H_2}{\partial x_k} & 0 & -1 & \ldots & 0 \\ \ldots \\ \dfrac{\partial H_{m - k}}{\partial x_1} & \ldots & \dfrac{\partial X_{m - k}}{x_k} & 0 & 0 & \ldots & -1 \end{pmatrix}$, где $m - k$ строчек и все они ЛНЗ.
                        
                    \item $2 \Rightarrow 1$:
                    
                        Из предыдущего пункта у нас есть система уравнение, для которой верно, что $\grad f_i(p)$~--- ЛНЗ, можно считать, что $\det \left( \dfrac{\partial f_i}{\partial x_{k + j}} (p) \right)_{i, j = 1 .. m - k} \neq 0$.
                        
                        По теореме о неявном отображении $\exists H : P \rightarrow Q$, где $P$~--- окрестность ($p_1$, $\ldots$, $p_k$), а $Q$~--- окрестность ($p_{k + 1}$, $\ldots$, $p_m$),
                        
                        что $\forall$ ($x_1$, $\ldots$, $x_k$) $\in P$ точка ($x_1$, $\ldots$, $x_k$, $H_1(x_1, \ldots, x_k)$, $H_2(x_1, \ldots, x_k)$, $\ldots$, $H_{m - k}(x_1, \ldots, x_k)$) удовлетворяет системе уравнений.
                        
                        $\Phi : P \rightarrow \RM$,
                        
                        $u \mapsto (u, H(u))$~--- параметризация нашего многообразия, $\left( P \times Q \right) \cap M$.
                        
                \end{itemize}
                
    \newpage
    
    \section{Следствие о двух параметризациях}
    
        $M \subset \RM$~--- $k$-мерное простое $C^r$-гладкое многообразие, $p \in M$, $U$~--- открытое в $M$, $p \in U$.
        
        $\Phi_1 : O_1 \subset \mathbb{R}^k \rightarrow U \cap M$,
        
        $\Phi_2 : O_2 \subset \mathbb{R}^k \rightarrow U \cap M$ (оба отображние ''на'' и даже гомеоморфизм)
        
        ($\phi_i \in C^r \left( O_i, \RM \right)$). Тогда существует диффеоморфизм $\psi : O_1 \rightarrow O_2$ и $\Phi_1 = \Phi_2 \circ \psi$.
        
        \subsection{Доказательство}
        
            Для случая, когда $\rang \Phi'_1(p)$ и $\rang \Phi'_2(p)$ на одном и том же наборе столбцов (во всех точках $O_1$ и $O_2$).
            
            Тогда $\Phi_1 \circ L$ и $\Phi_2 \circ L$~--- тоже диффеоморфизмы.
            
            Дальше всё очевидно, что $\Phi_1 = \Phi_2 \circ \left( L \circ \Phi_2 \right)^{-1} \circ \left( L \circ \Phi_1 \right)$.
            
    \newpage
    
    \section{Лемма о корректности определения касательного пространства}
    
        $\Phi : O \subset \mathbb{R}^k \rightarrow \RM$~--- $C^r$-параметризация $U(p) \cap M$, $p \in M$, $\Phi(t_0) = p$, $M$~--- простое $k$-мерное гладкое многообразие в $\RM$. Тогда образ оператора $\Phi'(t_0) : \mathbb{R}^k \rightarrow \RM$~--- это $k$-мерное подпространство в $\RM$, не зависящее от $\Phi$.
        
        \subsection{Доказательство}
        
            $\Phi$~--- параметризация, значит $\rang \Phi' = k$, значит образ $k$-мерный. Если есть параметризацция $\Phi_2$, можно считать, что существует диффеоморфизм $\psi$, что $\Phi_2 = \Phi \circ \psi$, и при этом $\Phi'_2 = \Phi' \cdot \psi'$, где $\psi'$~--- невырожденный, значит $\Phi'_2$ совпадает с $\Phi'$.
            
    \newpage
    
    \section{Касательное пространство в терминах векторов скорости гладких путей}
    
        $v \in Tp(M) \subset \RM \Longleftrightarrow$ существует гладкий путь $\gamma_v : [-1, 1] \rightarrow M$, $\gamma'(0) = v$ и $\gamma(0) = p$.
        
        \subsection{Доказательство}
        
            $\Phi$~--- параметризация в окрестности $P$, $\Phi(t_0) = p$.
            
            \begin{itemize}
            
                \item $\Leftarrow$
                
                    $\phi(t) = \Phi^{-1} ( \gamma (t) )$~--- соответствующий путь в $E$.
                    
                    Путь гладкий, значит $\gamma'(t) = \Phi(\phi(t))' = \Phi'(\phi(t)) \cdot \phi'(t)$, $\gamma'(0) = \Phi'(t_0)w$, что и требовалось доказать.
                    
                \item $\Rightarrow$
                
                    $v \in T_p(M) \rightarrow \exists w \in \mathbb{R}^k : \Phi'(t_0)w = v$.
                    
                    Рассмотрим путь $\gamma(t) = \Phi(t_0 + wt)$: $\gamma'(0) = \Phi'(t_0)w$, что и требовалось доказать.
                
            \end{itemize}
            
    \newpage
    
    \section{Касательное пространство к графику функции и к поверхности уровня}
    
        Касательное пространство к графику $f : O \subset \RM \rightarrow \mathbb{R}$, где $f \in C^1$ в точке $p = (x_0, f(x_0))$ задаётся уравнением
        
        $y - f(x_0) = f'_1(x_1) (x - x_1) + \ldots + f'm(x_m)(x - x_m)$.
        
        Касательное пространство к поверхности уровня функции $f : \mathbb{R}^3 \rightarrow \mathbb{R}$ задаётся уравнением
        
        $f'_x(x_0)(x - x_0) + f'_y(y_0)(y - y_0) + f'_z(z_0)(z - z_0) = 0$.
        
    \newpage
    
    \section{Необходимое условие относительного локального экстремума}
    
        $f : E \subset \mathbb{R}^{m + n} \rightarrow \mathbb{R}$,
        
        $\Phi : E \rightarrow \mathbb{R}^n$, $a \in E$ и $\Phi(a) = 0$~--- точка относительно локального экстремума.
        
        $\rang \Phi'(a) = n$, и $f$, $\Phi \in C^1 \left( E \right)$.
        
        Тогда $\exists \lambda = (\lambda_1, \ldots, \lambda_n) \in \mathbb{R}^n$, что
        
        $
            \begin{cases}
            
                f'(a) - \lambda \cdot \Phi'(a) = 0 \\
                
                \Phi(a) = 0
                
            \end{cases}
        $
        
        \subsection{Доказательство}
        
            Пусть $\rang \Phi'(a)$ реализован на столбцах $x_{m + 1}$, $\ldots$, $x_{m + n}$
            
            $a = (a_1, \ldots, a_m, a_{m + 1}, \ldots, a_{m + n}) = (a_x, a_y)$ и $\left( \frac{\partial \Phi}{\partial y} \right)$~--- невырожденная матрица $n \times n$.
            
            По теореме о неявном отображении $\varphi : U(a_x) \rightarrow V(a_y)$ и $\forall x \in U(a_x)$ $\Phi (X, \varphi(X) ) = 0$. Кстати, $U(x) \cap M_{\Phi}$~--- простое $m$-мерное многообразие. Тогда $a_x = (a_1, \ldots, a_m)$~--- точка локального экстремума для функции $g(x) = f(x, \varphi(x))$. Тогда
            
            \begin{equation*}
                \begin{cases}
                
                    f'_x(a) + f'_y(a)\varphi'(a_x) = 0 \\
                    
                    \Phi'_x(a) + \Phi'_y(a) \cdot \varphi'(a_x) = 0
                    
                \end{cases}
            \end{equation*}
            
            $\forall \lambda \in \mathbb{R}^n$ $\lambda \cdot \Phi'_x + \lambda \cdot \Phi'_y \cdot \varphi' = 0$
            
            $(f'_x - \lambda \Phi'_x) + (f'_y - \lambda \Phi'_y) \cdot \varphi' = 0$, подставляем $\lambda := f'_y(a) \cdot \left( \Phi'_y (a) \right)^{-1}$.
            
    \newpage
    
    \section{Вычисление нормы линейного оператора с помощью собственных чисел}
    
        $A \in \Lin (\RM, \mathbb{R}^n)$, тогда $\| A \| = \max \sqrt{\lambda}$, где $\lambda$~--- собственные числа $A^T A$.
        
        \subsection{Доказательство}
        
            $\| A \|^2 = \max | Ax |^2 = \max \left\langle Ax, Ax \right\rangle = \max \left\langle A^T Ax, x \right\rangle$.
    
    \newpage
    
    \section{Теорема Стокса--Зайдля о непрерывности предельной функций. Следствие для рядов}
    
        $f_n$, $f_0 : X \rightarrow \mathbb{R}$, $X$~--- метрическое пространство, $c \in X$, $f_n$~--- непрерывно в точке $c$. $f_n \rightrightarrows f_0$ на $X$. Тогда $f_0$~--- непрерывна в точке $c$.
        
        \subsection{Доказательство}
        
            $| f_0(x) - f_0(c) | \leq | f_0(x) - f_n(x) | + | f_n(x) - f_n(c) | + | f_n(c) - f_0(c) | < 3 \varepsilon$ (китайский эпсилон), поскольку по непрерывности из условия
            
            $\forall \varepsilon > 0 : \exists U(c) : \forall x \in U(c) : | f_0(x) - f_0(c) | < \varepsilon$.
            
        \subsection{Следствие для рядов}
        
            $u_n : X \rightarrow \mathbb{R}$ непрерывно в $x_0 \in X$, где $X$~--- метрическое пространство/
            
            $\sum u_n(x)$~--- равномерно сходится на $X$, $S(x) = \sum u_n(x)$. Тогда $S(x)$ непрерывно в $x_0$.
            
            \subsubsection{Доказательство}
        
                $S_n(x) \rightrightarrows S(x) \Rightarrow S(x)$~--- непрерывно в $x_0)$.
            
    \newpage
    
    \section{Метрика в пространстве непрерывных функций на компакте, его полнота}
    
        $X$~--- метрическое пространство, комактен. $f_1$, $f_2 : X \rightarrow \mathbb{R}$, $f_1$, $f_2$~--- непрерывен на $X$.
        
        $\rho(f_1, f_2) = \max\limits_{x \in X} | f_1(x) - f_2(x) |$~--- метрика в $C(x)$, тогда пространство $(C(x), \rho)$~--- полное.
        
        \subsection{Доказательство}
        
            $f_n \in C(x)$~--- фундаментальная последовательность
        
            $\forall \varepsilon > 0 : \exists N : \forall n, m > N : \max\limits_{x \in X} \left| f_n(x) - f_m(x) \right| < \varepsilon \Rightarrow \forall x \in X$.
            
            $f_n(x)$~--- фундаментальная вещественна последовательность, значит $\forall x \in X : \exists \lim\limits_{n \rightarrow +\infty} f_n(x) = f(x)$.
            
            $\forall \varepsilon > 0 : \exists N : \forall m, n > N : \forall x | f_n(x) - f_m(x) | < \varepsilon \Rightarrow | f_n - f(x) | < \varepsilon$
            
    \newpage
    
    \section{Теорема о предельном переходе под знаком интеграла. Следствие для рядов}
    
        $f_n \in C[a, b]$
        
        $f_n \rightrightarrows f$ на $[a, b]$.
        
        Тогда $\int\limits^b_a f_n \rightarrow \int\limits^b_a f$
        
        \subsection{Доказательство}
        
            $\left| \int\limits^b_a f_n - \int\limits^b_a f \right| \leq \int\limits^b_a | f_n - f | \leq \max\limits_{x \in [a, b]} | f_n(x) - f(x) | \cdot (b - a) \rightarrow 0$ ($a, b \in \mathbb{R}$, не в $\overline{\mathbb{R}}$)
            
        \subsection{Следствие для рядов}
        
            $u_n \in C[a, b]$
            
            $\sum u_n(x)$ равномерно сходится на $[a, b]$
            
            $S(x) = \sum\limits^{+\infty}_{n = 1} u_n(x)$, $x \in [a, b]$
            
            Тогда $\int\limits^b_a S(x) dx = \sum\limits^{+\infty}_{n = 1} \int\limits^b_a u_n(x) dx$
            
            ($\sum u_n$~--- равномерно сходится, $u_n$~--- непрерывно $\Rightarrow$ $S(x)$ непрерывно $\Rightarrow \int S(x)$ имеет смысл)
            
            \subsubsection{Доказательство}
            
                $\int\limits^b_a S_n(x)dx \rightarrow \int\limits^b_a S(x)dx$ по основной теореме,
                
                $\sum\limits^n_{k = 1} \int\limits^b_a u_k(x)dx \rightarrow \sum\limits^{+\infty}_{k = 1} \int\limits^b_a u_k(x)dx$.
                
    \newpage
    
    \section{Правило Лейбница дифференцирования интеграла по параметру}
    
        $f : [a, b] \times [c, d] \rightarrow \mathbb{R}$
            
        $f$, $f'_y$~--- непрерывны на $[a, b] \times [c, d]$, $\Phi(y) = \int\limits^b_a f(x, y) dx$
            
        Тогда $\Phi$~--- дифференцируема на $[c, d]$ и $\Phi'(y) = \int\limits^b_a f'_y (x, y) dx$
            
        \subsection{Доказательство} 
            
            $\dfrac{\Phi \left( Y + \frac{1}{n} \right) - \Phi(y)}{\frac{1}{n}} = \int\limits^b_a \dfrac{f \left( x, y + \frac{1}{n} \right) - f(x, y)}{\frac{1}{n}} dx = \int\limits^b_a f'_y \left(x, y + \frac{\Theta}{n} \right) dx$, что есть $\int\limits^b_a g_n dx$.
            
            $g_n(x, y) \rightrightarrows f'_y (x, y)$ для $x \in [a, b]$.
            
            $\Theta = a_x$, $0 \leq a_x \leq 1$.
            
    \newpage
    
    \section{Теорема о предельном переходе под знаком производной. Дифференцирование функционального ряда}
    
        $f_n \in C^1 \langle a, b \rangle$ и $f_n \rightarrow f_0$ поточечно на $\langle a, b \rangle$, $f'_n \rightrightarrows \varphi$ на $\langle a, b \rangle$. Тогда 
        
        \begin{enumerate}
        
            \item $f_0 \in C^1 \langle a, b \rangle$
            
            \item $f'_0 = \varphi$ на $\langle a, b \rangle$
            
        \end{enumerate}
        
        \subsection{Доказательство}
        
            $x_0$, $x_1 \in \langle a, b \rangle$, $f'_n \rightrightarrows \varphi$ на $[x_0, x_1]$, $\int\limits^{x_1}_{x_0} f'_n \rightarrow \int\limits^{x_1}_{x_0} \varphi$
            
            $f_n(x_0) - f_n(x_0) \xrightarrow[n \rightarrow +\infty]{} \int\limits^{x_1}_{x_0} \varphi$, и $f_n(x_1) - f_n(x_0) \rightarrow f_0(x_1) - f_0(x_0)$, значит
            
            $\int\limits^{x_1}_{x_0} \varphi = f_0(x_1) - f_0(x_0)$, $f_0$~--- первообразная для $\varphi$, $\varphi$~--- непрерывна, значит $f') - \varphi$.
            
        \subsection{Дифференцирование функционального ряда}
        
            $u_n \in C^1 \left( \langle a, b \rangle \right)$
            
            \begin{enumerate}
            
                \item $\sum u_n(x) = S(x)$ $x \in \langle a, b \rangle$ (поточечная сходимость)
                
                \item $\sum u'_n(x) = \varphi(x)$ равномерно сходится при $x \in \langle a, b \rangle$.
            
            \end{enumerate}
            
            Тогда
                
            \begin{enumerate}
                
                \item $S(x) \in C^1 \left( \langle a, b \rangle \right)$
                    
                \item $S'(x) = \varphi(x)$ при $x \in \langle a, b \rangle$
                    
            \end{enumerate}
                
            Т.е. $\left( \sum\limits^{+\infty}_{n = 1} u_n(x) \right)' = \sum\limits^{+\infty}_{n = 1} u'_n(x)$
            
        \subsubsection{Доказательство}
            
            Следует из основной теоремы.
                
            $f_n \leftrightarrow S_n$ и $f_0 \leftrightarrow S$.
                
            $f_n(x) \rightarrow f_0(x)$ и $f'_n \rightrightarrows \varphi$ и $\sum\limits^n_{k = 1} u'_k(x) = \left( \sum\limits^n_{k = 1} u_k(x) \right)' = f'_n$
                
    \newpage
    
    \section{Признак Вейерштрасса равномерной сходимости функционального ряда}
    
        $\sum u_n$ и $u_n : X \rightarrow \mathbb{R}$. Также пусть существует вещественная последовательность $c_n$:
        
        \begin{enumerate}
        
            \item $| u_n(x) | \leq c_n$ $\forall x \in X$;
            
            \item $\sum c_n$ сходится.
            
        \end{enumerate}
        
        Тогда $\sum u_n(x)$~--- равномерно сходится на $X$
        
        \subsection{Доказательство}
        
            Равномерно сходится тогда и только тогда $R_n \rightrightarrows 0$
            
            $\sup\limits_{x \in X} \left| \sum\limits^{+\infty}_{k = n} u_k(x) \right| \leq \sum\limits^{+\infty}_{k = n} c_k \xrightarrow[n \rightarrow +\infty]{} 0$ как остаток сходящегося ряда.
        
    \newpage
    
    \section{Дифференцируемость гамма функции}
    
        $\Gamma(x)$ бесконечно дифференцируется на $(0, +\infty)$.
        
        \subsection{Доказательство}
        
            $\dfrac{1}{\Gamma(x)} = x e^{\gamma x} \prod\limits^{+\infty}_{k = 1} \left( 1 + \frac{x}{k} \right) e^{-\frac{x}{k}}$.
            
            $-\ln \Gamma(x) = \ln x + \gamma x + \sum\limits^{+\infty}_{k = 1} \left( \ln \left( 1 + \frac{x}{k} \right) - \frac{x}{k} \right)$, обозначим за $u_k = \left( \ln \left( 1 + \frac{x}{k} \right) - \frac{x}{k} \right)$.
            
            $u'_k(x) = \frac{1}{x + k} - \frac{1}{k} = -\frac{x}{k(k + x)}$.
            
            $| u'_k(x) | \leq \frac{M}{k (k + M)}$, $\sum \frac{M}{k (k + M)}$~--- сходится по признаку Вейерштрасса.
            
            $\sum u'_k(x)$ равномерно сходится при $x \in (0, M)$, где $M$~--- какое угодно.
            
            $- \dfrac{\Gamma'(x)}{\Gamma(x)} = \frac{1}{x} + \gamma - \sum \frac{x}{k(k + x)}$.
            
    \newpage
    
    \section{Теорема о предельном переходе в суммах}
    
        $u_n : E \subset X \rightarrow \mathbb{R}$, $x_0$~--- предельная точка $E$, $X$~--- метрическое пространство.
        
        \begin{enumerate}
        
            \item $\forall n : \exists \lim\limits_{x \rightarrow x_0} u_n(x) = a_n$;
            
            \item $\sum u_n(x)$ равномерно сходится на $E$.
        
        \end{enumerate}
        
        Тогда
        
        \begin{enumerate}
        
            \item $\sum a_n$~--- сходится;
            
            \item $\sum a_n = \lim\limits_{x \rightarrow x_0} \left( \sum\limits^{+\infty}_{n = 1} u_n(x) \right)$.
            
        \end{enumerate}
        
        \subsection{Доказательство}
        
            \begin{enumerate}
            
                \item $\sum a_n$~--- сходится
                
                    $S_n(x) = \sum\limits^n_{k = 1} u_k(x)$, $S^a_n = \sum\limits^n_{k = 1} a_k$.
                    
                    Достаточно проверить, что последовательность $S^a_n$ фундаментальная.
                    
                    $\left| S^a_{n + p} - S^a_n \right| \leq \left| S^a_{n + p} - S_{n + p} (x) \right| + \left| S_{n + p}(x) - S_n(x) \right| + \left| S_n(x) - S^a_n \right| < \varepsilon$
                    
                \item Сводим к предыдущей теореме
                
                    $\widetilde{u_n}(x)$ := $\left[
                        \begin{gathered} 
                            u_n(x), x \neq x_0, x \in E 
                            \\ 
                            a_n, x = x_0 \hfill
                        \end{gathered}
                        \right.$
                    
                    $\widetilde{u_n}$~--- непрерывна в точке $x_0$. Остаётся только проверить, что $\sum \widetilde{u_n}(x)$ равномерно сходится в $E \cup \left\{ x_0 \right\}$
                    
                    $\sup\limits_{x \in E \cup \left\{ x_0 \right\}} \left| \sum\limits^{+\infty}_{k = n} \widetilde{u_k}(x) \right| \leq \sup\limits_{x \in E} \left| \sum\limits^{+\infty}_{k = n} u_k(x) \right| \rightarrow 0$
                    
            \end{enumerate}
        
    \newpage
    
    \section{Теорема о перестановке двух предельных переходов}
    
        $f_n : E \subset X \rightarrow \mathbb{R}$, $x_0$~--- предельная точка $E$, и
        
        \begin{enumerate}
        
            \item $f_n \rightrightarrows S(x)$ при $n \rightarrow +\infty$ на $E$;
            
            \item $f_n(x) \xrightarrow[x \rightarrow x_0]{} A_n$
            
        \end{enumerate}
        
        Тогда
        
        \begin{enumerate}
        
            \item $\exists \lim\limits_{n \rightarrow +\infty} A_n = A \in \mathbb{R}$
            
            \item $S(x) \xrightarrow[x \rightarrow x_0]{} A$
            
        \end{enumerate}
        
        \subsection{Доказательство}
        
            $\sum\limits^n_{k = 1} u_k = f_n$, $a_k := A_k - A_{k - 1}$
            
            Тогда $\sum a_n$~--- сходится $\Rightarrow \exists \sum \left( A_k - A_{k - 1} \right) \Leftrightarrow \exists \lim\limits_{n \rightarrow +\infty} A_n$
            
            $f(x, y) \rightrightarrows f(x)$ при $y \rightarrow y_0$ на множестве $E$, т.е.
            
            $\forall \varepsilon > 0 : \exists > 0 : \forall y : 0 < | y - y_0 | < \delta : \forall x \in E : | f(x, y) - f(x, y_0) | < \varepsilon$
            
    \newpage
    
    \section{Признак Дирихле равномерной сходимости функционального ряда}
    
        $\sum a_n(x) b_n(x)$, $x \in X$.
        
        \begin{enumerate}
        
            \item $\exists C_a : \forall N : \forall x \in X : \left| \sum\limits^N_{n = 1} a_n(x) \right| \leq C_a$, 
            
                частичные суммы ряда $\sum a_n(x)$ равномерно ограничены.
            
            \item $b_n \rightrightarrows 0$ при $n \rightarrow +\infty$ на множестве $X$, $\forall x : b_n(x)$~--- монотонная. Тогда $\sum a_n(x) b_n(x)$ равномерно сходится на $X$.
            
        \end{enumerate}
        
        \subsection{Доказательство}
        
            $\sum\limits_{N \leq k \leq M} a_k b_k = A_M b_M - A_{N - 1} b_{N - 1} + \sum\limits^{M - 1}_{k = N} (b_k - b_{k + 1}) A_k$
            
            $\left| \sum\limits^M_{k = N} a_k b_k \right| \leq |A_M b_M| + |A_{N - 1} b_N| + \left| \sum (b_k - b_{k + 1} ) A_k \right| \leq C_A \left( |b_M| + |b_N| \right) + \sum(b_k - b_{k + 1})A_k \leq C_a \left( | b_M | + | b_N | + \sum^{M - 1}_{k = N} (b_k - b_{k + 1}) \right) \leq C_a \left( |b_M| + |b_N| + |b_M| + |b_N| \right) \rightarrow c$
            
            \begin{enumerate}
            
                \item $\sum\limits^{+\infty}_{n = 1} a_n(x)$~--- равномерно сходится $x \in X$;
                
                \item $\exists C_B : \forall x \forall n : |b_n(x)| \leq C_m$ при каждом $x$ $b_n(x)$ монотонна. $\sum a_n b_n$ равномерно сходится/
                
            \end{enumerate}
            
    \newpage
    
    \section{Теорема о круге сходимости степенного ряда}
    
        $\sum a_n (z - z_0)^n$, тогда выполнено одно из трёх условий:
        
        \begin{enumerate}
        
            \item ряд сходится только при $z = z_0$;
            
            \item ряд сходится при любых $z \in \mathbb{C}$;
            
            \item $\exists R \in (0, +\infty)$ такое, что при $|z - z_0| < R$ абсолютно сходится, при $|z - z_0| > R$ расходится, при $|z - z_0| = R$ может как сходится, так и расходится.
            
        \end{enumerate}
        
        \subsection{доказательство}
        
            Изучим $\sum a_n (z - z_0)^n$ на абсолютная сходимость.
            
            $\overline{\lim\limits_{n \rightarrow +\infty}} \sqrt[n]{|a_n||z - z_0|^n} = \overline{\lim} |z - z_0| \sqrt[n]{|a_n|} = |z - z_0| \sqrt[n]{|a_n|}$
            
            \begin{enumerate}
            
                \item $\overline{\lim} \sqrt[n]{a_n} = +\infty$, тогда при $z = z_0$ ряд абсолютно сходится, при $z \neq z_0$ ряд расходится;
                
                \item $\overline{\lim} \sqrt[n]{a_n} = 0$, тогда при любых $z$ ряд сходится абсолютно;
                
                \item $\overline{\lim} \sqrt[n]{a_n}$ конечен, тогда при $|z - z_0| < \dfrac{1}{\overline{\lim}\sqrt[n]{a_n}}$~--- сходится, при $|z - z_0| > \dfrac{1}{\overline{\lim}\sqrt[n]{a_n}}$~--- расходится.
                
                    Тогда обозначим $R = \dfrac{1}{\overline{\lim}\sqrt[n]{|a_n|}}$~--- формула Адомара.
                
            \end{enumerate}
            
            Множество сходимости степенного ряда~--- это открытый круг радиуса $R$ и некоторые точки на окружности.
            
    \newpage
    
    \section{Теорема о непрерывности степенного ряда}
    
        $\sum a_n(z - z_0)^n$, $0 < R \leq +\infty$. Тогда 
        
        \begin{enumerate}
        
            \item $0 < r < R$~--- тогда ряд равномерно сходится на $\overline{B(z_0, r)}$;
            
            \item $f(z) = \sum a_n (z - z_0)^n$~--- непрерывен в $B(z_0, R)$.
        
        \end{enumerate}
        
        \subsection{Доказательство}
        
            \begin{enumerate}
            
                \item По признаку Вейерштрасса $\left| a_n (z - z_0)^n \right| \leq | a_n | \cdot r^n$, ряд $\sum | a_n | r^n$~--- абсолютно сходится;
                
                \item Очевидно из предыдущего пункта.
                
            \end{enumerate}
            
    \newpage
    
    \section{Теорема о дифференцировании степенного ряда. Следствие об интегрировании. Пример}
    
        Обозначим за $A$ ряд $\sum\limits^{+\infty}_{n = 0} a_n (z - z_0)^n$ и за $A'$ ряд $\sum\limits^{+\infty}_{n = 1} n a_n (z - z_0)^{n - 1}$, $0 < R \leq +\infty$~--- радиус сходимости для $(A)$. Тогда
        
        \begin{enumerate}
        
            \item $(A')$ имеет тот же радиус сходимости $R$;
            
            \item Пусть $f(z) = \sum a_n (z - z_0)^n$, $z \in B(z_0, R)$. Тогда $\forall z \in B(z_0, R)$ $f$~--- дифференцируема и $f'(z) = \sum n a_n (z - z_0)^{n - 1}$.
            
        \end{enumerate}
        
        \subsection{Доказательство}
        
            \begin{enumerate}
            
                \item $\sum \alpha_n x^n$ и $\sum \alpha_n x^{n + 1}$ имеют одинаковый радиус сходимости, т.к. $x \cdot S_N(x) = \widetilde{S_N(x)}$. Пределы этих сумм существуют для одинаковых $x$, значит и радиус сходимости один и тот же.
                
                    $R_{A'} = \dfrac{1}{\overline{\lim}\sqrt[n]{n a_n}} = \dfrac{1}{\overline{\lim}\sqrt[n]{n}\sqrt[n]{a_n}} = R$.
                    
                \item $a \in B(z_0, R)$, проверим, что существует $f'(a)$. Возьмём $r < R$ и $a \in B(z_0, r)$. Также пусть $w = z - z_0$ и $w_0 = a - z_0$, $| z - z_0 | < r$ и $| a - z_0 | < r$, тогда
                
                    $\lim \frac{f(z) - f(a)}{z - a} = \sum a_n \frac{(z - z_0)^n - (a - z_0)^n}{(z - z_0) - (a - z_0)} = \sum a_n \frac{w^n - w_0^n}{w - w_0}$ и 
                    
                    $\left| a_n \frac{w^n - w_0^n}{w - w_0} \right| \leq | a_n | n r^{n - 1}$.
                    
                    Заметим, что $\sum n | a_n | r^{n - 1}$ сходится, т.к. ряд $(A')$ при $z = z_0 + r$ сходится абсолютно по признаку Вейерштрасса, ряд равномерно сходится в круге $B(z_0, r)$.
                    
                    $\lim\limits_{z \rightarrow a} \frac{f(z) - f(z_0)}{z - z_0} = \lim \sum\limits^{+\infty}_{n = 0} \ldots = \sum\limits^{+\infty}{n = 1} \lim\limits_{z \rightarrow a} \frac{(z - z_0)^n - (a - z_0)^n}{z - a} = \sum\limits^{+\infty}_{n = 1} a_n n (z - z_0)^{n - 1}$.
                    
            \end{enumerate}
            
        \subsection{Следствие об интегрировании}
        
            $f(x) = \sum a_n (x - x_0)^n$, $a_n \in \mathbb{R}$, $x_0 \in \mathbb{R}$, $x$~--- тоже вещественное и лежит в $(x_0 - R, x_0 + R)$.
            
            Тогда при почленном интегрировании $\sum a_n \frac{(x - x_0)^{n + 1}}{n + 1}$~--- ряд имеет тот же радиус сходимости и к тому же $\int\limits^x_{x_0} \left( \sum\limits^{+\infty}_{n = 0} a_n (x - x_0)^n \right) dx = \sum\limits^{+\infty}_{n = 0} \frac{a_n}{n + 1} (x - x_0)^{n + 1}$.
            
        \subsection{Пример}
        
            Разложить $\arcctg x$ в степенной ряд в окрестности $x_0 = 0$ (это же ряд Тейлора)
            
            $(\arcctg x)' = -\frac{1}{1 + x^2} = -(1 - x^2 + x^4 - x^6 + \ldots) = -1 + x^2 - x^4 + x^6 + \ldots$ ($|x| < 1$)
            
            $\arcctg x = \frac{\pi}{2}-x + \frac{x^3}{3} - \frac{x^5}{5} + \ldots$. (не забудем, что при возврате к первообразной не надо забывать про константу).

    \newpage
    
    \section{Свойства экспоненты}
    
        Обозначим $\exp(z) = \sum\limits^{+\infty}_{n = 0} \frac{z^n}{n!}$, $R = +\infty$, сходится при всех $z \in \mathbb{C}$.
        
        \begin{enumerate}
        
            \item $\exp(0) = 1$;
            
            \item $(\exp z)' = \exp z$
            
                $\lim\limits_{z \rightarrow 0} \frac{e^z - 1}{z} = (e^z)' \bigg|_{z = 0} = 1$;
                
            \item $\overline{\exp(z)} = \exp(\overline{z})$ комплексное, $\overline{\sum \frac{z^n}{n}} = \sum \frac{\overline{z}^n}{n}$;
            
            \item $\exp (z + w) = \exp(z) \cdot \exp(w)$
            
                $\exp (z + w) = \sum\limits^{+\infty}_{n = 0} \frac{(z + w)^n}{n!} = \sum\limits^{+\infty}_{n = 0} \left( \sum\limits^n_{k = 0} \frac{z^k}{k!} \frac{w^{n - 1}}{(n - k)!} \right) = \left( \sum \frac{z^k}{k!} \right) \left( \sum \frac{w^k}{k!} \right)$
                
        \end{enumerate}
        
        \subsection{Следствие}
        
            $\forall z \in \mathbb{C}$ $\exp(z) \neq 0$.
            
    \newpage
    
    \section{Метод Абеля суммирования рядов. Следствие}
    
        $\sum c_n$~--- сходящийся ряд, $f(x) = \sum c_n x^n$, $-1 < x < 1$ ($\Leftrightarrow R \geq 1$). Тогда $\sum c_n = \lim\limits_{x \rightarrow 1 - 0} f(x)$.
            
        \subsection{Доказательство}
        
            При $x \in (0, 1)$, $\sum c_n x^n$~--- сходится по признаку Абеля,
            
            $\sum a_n b_n$, $\sum a_n$~--- сходится, $b_n$~--- монотонно ограниченная, что чему сопоставить очевидно. Осталось проверить, что $\sum c_n x^n$ непрерывен на $[0, 1]$, т.е. равномерную сходимость $\sum c_n x^n$ на $[0, 1]$.
            
            $\sum a_n(x)$~--- равномерно сходится, $b_n$~--- монотонная при каждом фиксированном $x$, $\exists C_b : \forall n : \forall x : | b_n(x) | \leq C_b$.
            
        \subsection{Следствие}
        
            $\sum a_n = A$, $\sum b_n = B$, $c_n := a_0 b_n + a_1 b_{n - 1} + \ldots + a_n b_0$, известно, что $\sum c_n = C$. Тогда $A \cdot B = C$.
            
            \subsubsection{Доказательство}
            
                $f(x) = \sum a_n x^n$, $g(x) = \sum b_n x^n$, $h(x) = \sum c_n x^n$, $x \in [0, 1]$.
                
                $x < 1$ ряды для $f$ и $g$ абсолютно сходится, значит $f(x) \cdot g(x) = h(x)$ при $x \rightarrow 1$.
                
    \newpage
    
    \section{Единственность разложения функции в ряд}
    
        $f$ единственным образом раскладывается в степенной ряд в окрестности $x_0$ (если можно, конечно, разложить его).
        
        \subsection{Доказательство}
        
            Потому что $a_n := \dfrac{f^{(n)}(x_0)}{n!}$
            
            $f(x) = a_0 + a_1(x - x_0) + a_2(x - x_0)^2 + \ldots \Rightarrow f \in C^{+\infty} \left( U(x_0) \right)$.
            
            $x := x_0 \Rightarrow a_0 = f(x_0)$.
            
            $f'(x) = a_1 + 2 a_2 (x - x_0) + 3 a_3 (x - x_0)^2 + \ldots$, $x := x_0 \Rightarrow a_1 = f'(x_0)$ и $a_2 = \frac{f''(x_0)}{2!}$ и т.д.
            
    \newpage
    
    \section{Разложение бинома в ряд Тейлора}
    
        $\sigma \in \mathbb{R}$, тогда при $|x| < 1$
        
        $(1 + x)^{\sigma} = 1 + \sigma x + \frac{\sigma (\sigma - 1)}{2}x^2 + \ldots + \frac{\sigma (\sigma - 1) \ldots (\sigma - n + 1)}{n!} x^n + \ldots$.
        
        \subsection{Доказательство}
        
            $S'(x)(1 + x) = \sigma S(x)$
            
            $f(x) = \dfrac{S(x)}{(1 + x)^{\sigma}} = const$
            
            $f' = \dfrac{S'(x)}{(1 + x)^{\sigma}} - \dfrac{\sigma S(x)}{(1 + x)^{\sigma + 1}} = \dfrac{0}{(1 + x)^{\sigma + 1}} \Rightarrow \dfrac{S(x)}{(1 + x)^{\sigma}} = const$
            
            $f(0) = 1 \rightarrow S(x) = (1 + x)^{\sigma}$.
            
    \newpage
    
    \section{Теорема о разложимости функции в ряд Тейлора}
    
        $f \in C^{\infty} \left( [x_0 - h, x_0 + h] \right)$. Тогда эквивалентны следующие утверждения:
        
        \begin{enumerate}
        
            \item $f$ раскладывается в ряд Тейлора в окрестности $x_0$;
            
            \item $\exists \delta, C, A > 0 : \forall n : \left| f^{(n)}(x) \right| < C \cdot A^n \cdot n!$ при $|x - x_0| < \delta$.
            
        \end{enumerate}
        
        \subsection{Доказательство}
        
            \begin{itemize}
            
                \item $1 \Leftarrow 2$
                
                    Оценим остаток в форме Лагранжа $f(x) = \sum\limits^{n - 1}_{k = 0} \dfrac{f^{(k)}(x_0)}{k!} (x - x_0)^k + \dfrac{f^{(n)}(\overline{x})}{n!} (x - x_0)^n$
                    
                    $\left| \dfrac{f^{(n)}(\overline{x})}{n!} (x - x_0)^n \right| \leq \dfrac{C A^n n!}{n!} |x - x_0|^n \rightarrow 0$ при $|A(x - x_0)| < 1$ и $|x - x_0| < \frac{1}{n}$. 
                    
                    Таким образом, $|x - x_0| < \min \left( \frac{1}{A}, \delta \right)$, $r_n \rightarrow 0$.
                    
                \item $1 \Rightarrow 2$
                
                    $f(x) = \sum\limits^{+\infty}_{n = 0} \dfrac{f^{(n)}(x_0)}{n!} (x - x_0)^n$. Пусть при $x = x_1 \neq x_0$ ряд сходится.
                    
                    $\dfrac{f^{(n)}(x_0)}{n!} (x_1 - x_0)^n \rightarrow 0$, т.е. меньше $C_1$ по модулю.
                    
                    $\left| f^{(n)}(x_0) \right| \leq C_1 \cdot n! \cdot \dfrac{1}{| x_1 - x_0 |^n} B^n$
                    
                    $f^{(m)}(x) = \sum\limits^{+\infty}_{n = m} \dfrac{f^{(n)}(x_0)}{(n - m)!} (x_0) (x - x_0)^{n - m}$
                    
                    $\left| f^{(m)}(x) \right| \leq \sum\limits^{+\infty}_{n = m} \dfrac{|f^{(n)}(x_0)}{(n - m)!} |x - x_0|^{n - m} \leq \sum \dfrac{C_1 B^n n!}{(n - m)!} |x - x_0|^{n - m} = C_1 B^n \sum\limits^{+\infty}_{n = m} n (n - 1) \ldots (n - m + 1) \left| B(x - x_0) \right|^{n - m} = C_1 \cdot \dfrac{m! B^m}{| 1 - \left( B(x - x_0) \right)}^{(m + 1)} \leq C_1 m! B^m 2^{m + 1} = (2 C_1) m! (2 B)^m$.
                    
            \end{itemize}
            
    \newpage
    
    \section{Теорема Коши о перманентности метода средних арифметических}
        
        \subsection{Дополнительное определение}
        
            $\sum\limits^{+\infty}_{n = 0} a_n$, $S_n = a_0 + a_1 + \ldots + a_n$.
            
            $\sigma_n = \frac{1}{n + 1} \left( S_0 + S_1 + \ldots + S_n \right)$.
            
            Если существует $\lim\limits_{n \rightarrow +\infty} \sigma_n = S$, то $S$ называется суммой ряда $\sum a_n$ в смысле метода средних арифметических (или по Чезаро).
        
        \subsection{Формулировка}
        
            $\sum a_n = S \Rightarrow \sum a_n = S$ в смысле метода средних арифметических.
        
        \subsection{Доказательство}
        
            $\forall \varepsilon > 0 : \exists N_1 > 0 : \forall n > N_1 : \left| S_n - S \right| < \varepsilon$
            
            $\sigma_n - S = \frac{1}{n + 1} \sum\limits^n_{i = 0} \left( S_i - S \right)$, $\left| \sigma_n - S \right| \leq \frac{1}{n + 1} \sum\limits^n_{i = 0} \left| S_i - S \right| = \dfrac{\sum\limits^{N_1}_{i = 0} \left( S_i - S \right)}{n + 1} + \dfrac{\sum\limits^n_{i = N_1 + 1} \left| S_i - S \right|}{n + 1} < 2 \varepsilon$.
            
    \newpage
    
    \section{Простейшие свойства интеграла векторного поля по кусочно-гладкому пути}
    
        \begin{enumerate}
        
            \item Линейность по полю:
            
                $\forall \alpha, \beta \in \mathbb{R}$, $U$, $V$~--- векторные поля, тогда
                
                $I \left( \alpha U + \beta V, \gamma \right) = \alpha I \left(U, \gamma \right) + \beta I \left( V, \gamma \right)$.
                
            \item Аддитивность при дроблении пути:
            
                $\gamma : [a, b] \rightarrow \RM$, $a < c < b$,
                
                $\gamma_1 := \gamma \big|_{[a, c]}$, $\gamma_2 := \gamma \big|_{[c, b]}$ и
                
                $I \left( V, \gamma \right) = I \left( V, \gamma_1 \right) + I \left( V, \gamma_2 \right)$.
                
            \item Замена параметра:
            
                $\varphi : [p, q] \rightarrow [a, b]$, сюрьекция, $\varphi \in C^1 \left( [p, q] \right)$, $\varphi(p) = a$, $\varphi(q) = b$,
                
                $\gamma : [a, b] \rightarrow \RM$, $\widetilde{\gamma} = \gamma \circ \varphi$, $\widetilde{\gamma}(s) = \gamma (\varphi(s))$.
                
                $I \left( V, \gamma \right) = I \left( V, \overline{\gamma} \right)$.
                
            \item $\gamma_1 : [a, b] \rightarrow \RM$ $\gamma_2 : [c, d] \rightarrow \RM$~--- гладкие пути,
            
                $\gamma_1(b) = \gamma_2(c) \Rightarrow \gamma = \gamma_2 \gamma_1$~--- кусочно-гладкий путь (в точке $b$ путь $\gamma$ может быть и не гладким).
                
                $\gamma(t) = \begin{cases}
                    \gamma_1(t), t \in [a, b] \\
                    \gamma_2(t - b + c), t \in [b, b + d - c]
                \end{cases}$
                
                Тогда $I \left( V, \gamma \right) = I \left( V, \gamma_1 \right) + I \left( V, \gamma_2 \right)$.
            
            \item $\gamma : [a, b] \rightarrow \RM$,
            
                $\gamma^- (t) = \gamma(a + b - t)$, $t \in [a, b]$. Тогда $I \left( V, \gamma^- \right) = - I \left( V, \gamma \right)$.
                
            \item Оценка интеграла по пути:
            
                $\gamma : [a, b] \rightarrow \RM$, $L := \gamma \left( [a, b] \right)$~--- носитель пути. Тогда
                
                $\left| I \left( V, \gamma \right) \right| \leq \max\limits_{x \in L} \left| V(x) \right| \cdot l (\gamma)$.
                
        \end{enumerate}
        
        \subsection{Доказательство}
        
            \begin{enumerate}
            
                \item Из определения в силу линейности скалярного произведения;
                
                \item $\int\limits^b_a = \int\limits^c_a + \int\limits^b_c$;
                
                \item $I \left( V, \gamma \right) = \int\limits^b_a V_1 \left( \gamma(t) \right) \gamma'_1 + \ldots + V_m \left( \gamma (t) \right) \gamma'_m dt = \int\limits^q_p \left( V_1 \left( \widetilde{\gamma}(s) \right) \gamma'_1 \left( \varphi(s) \right) + \ldots + V_m \left( \widetilde{\gamma}(s) \right) \gamma'_m \left( \varphi(s) \right) \right) \varphi'(s) ds = I \left( V, \widetilde{\gamma} \right)$;
                
                \item $\int\limits^{b + d - c}_{a} \langle V (\gamma(t)), \gamma'(t) \rangle dt = \int\limits^b_a + \int\limits^{b + d - c}_b = \int\limits^b_a + \int\limits^d_c \langle V (\gamma_2 (\tau) ), \gamma'_2(\tau) \rangle d \tau$;
                
                \item $I \left(V, \gamma^- \right) = \int\limits^b_a \langle V \left( \gamma(a + b - t) \right) \cdot \left( - \gamma'(a + b - t) \right) \rangle dt = - \int\limits^b_a \langle V, \left( \gamma(\tau) \right), \gamma'(\tau) (-d \tau) \rangle = -I \left(v, \gamma \right)$;
                
                \item $\left| \int\limits^b_a \langle V(\gamma), \gamma' \rangle dt \right| \leq \int\limits^b_a \left| \langle V, \gamma' \rangle \right| dt \leq \int\limits^b_a \left| V \left( \gamma(t) \right) \right| | \gamma'(t) | dt \leq \max\limits-{x \in L} \left| V(x) \right| \cdot \int\limits^b_a | \gamma'(t) | dt$.
                
            \end{enumerate}
            
    \newpage
    
    \section{Обобщенная формула Ньютона--Лейбница}
    
        $V : O \subset \RM \rightarrow \RM$, потенциальное векторное поле, $f$~--- потенциал, $\gamma [a, b] \rightarrow O$~--- кусочно-гладкий путь, $\gamma(a) = A$, $\gamma(b) = B$. Тогда
        
        $\int\limits_{\gamma} V_1 dx_1 + \ldots + V_m dx_m = f(B) - f(A)$.
        
        \subsection{Доказательство}
        
            \begin{enumerate}
            
                \item $\gamma$~--- гладкий, $\phi(t) = f(\gamma(t))$, $\phi' = f' \gamma' = \langle \grad f, \gamma' \rangle = \langle V \left( \gamma(t) \right), \gamma'(t) \rangle = f(B) - f(A)$.
                
                \item кусочно-гладкий
                
                    $I \left( V, \gamma \right) = \sum\limits^n_{k = 1} \int\limits^{t_k}_{t_{k - 1}} \ldots = \sum\limits^n_{k = 1} f \left( \gamma (t_k) \right) - f \left( \gamma ( t_{k - 1} ) \right) = f(\gamma(t_n)) - f(\gamma(t_0)) = f(B) - f(A)$.
                    
            \end{enumerate}
            
    \newpage
    
    \section{Характеризация потенциальных векторных полей в терминах интегралов}
    
        $V$~--- векторное поле в $O$. Тогда эквивалентны следующие утверждения:
        
        \begin{enumerate}
        
            \item $V$~--- потенциальное;
            
            \item Интеграл $\int\limits_{\gamma} V_1 dx_1 + \ldots + V_m dx_m$ не зависит от пути в $O$;
            
            \item Для любого кусочно-гладкого замкнутого пути верно, что $\int\limits_{\gamma} V_1 dx_1 + \ldots + V_m dx_m = 0$.
            
        \end{enumerate}
            
        \subsection{Доказательство}
        
            \begin{itemize}
            
                \item $1 \Rightarrow 2$~--- формула Ньютона-Лейбница;
                
                \item $2 \Rightarrow 3$~--- очевидно;
                
                \item $3 \Rightarrow 2$~--- очевидно;
                
                \item $2 \Rightarrow 1$ фиксируем $A \in O$, $\forall x \in O$ фиксируем кусочно-гладкий путь $\gamma_x$, $f(x) := \int\limits_{\gamma_x} V_1 dx_1 + \ldots + V_m dx_m$. Надо проверить, что $f$~--- потенциал.
                
                    Достаточно проверить, что $f'_{x_1}(x) = V_1(x)$ при всех $x$.
                    
                    $\gamma'_0 = (h, 0, \ldots, 0)$
                    
                    $f(x + he_1) - f(x) = \int\limits_{\gamma_0} V_ dx_1 + \ldots + V_m dx_m = \int\limits^1_0 V_1(x_1 + th, \ldots, x_m) h dt = V_1(x_1 + \alpha h, x_2, \ldots, x_m) h (1 - \alpha) \rightarrow V_1(x_1, \ldots, x_m)$.
                    
            \end{itemize}
            
    \newpage
    
    \section{Необходимое условие потенциальности гладкого поля. Лемма Пуанкаре}
    
        \subsection{Необходимое условие потенциальности гладкого поля}
        
            $V$~--- гладкое потенциальное векторное поле в $O \subset \RM$, тогда $\forall x \in O$ и $\forall k, j$ $(1 \leq k, j \leq m)$ верно $\frac{\partial v_k}{\partial x_j} = \frac{\partial v_j}{\partial x_k}$.
            
        \subsection{Лемма Пуанкаре}
        
            $O \subset \RM$~--- выпуклое, $V : O \rightarrow \RM$, $V \in C^1(O)$ и верно $\forall k$, $l : \frac{\partial v_k}{\partial x_l} = \frac{\partial v_l}{\partial x_k}$. Тогда $V$~--- потенциально.
            
            \subsubsection{Доказательство}
            
                $A \in O$, $\gamma_x : [0, 1] \rightarrow O$, $\gamma_v(t) = A + t(x - A)$, $(\gamma_x)' = x - A$.
                
                $f(x) := \int\limits_{\gamma_x} \sum v_i dx_i = \int\limits^1_0 \sum v_i \left( A + t \left( x - A \right) \right) \left( x_i - A_i \right) dt$, $I(x) = \int\limits^b_a f(c, x)dt$ и $I'(x) = \int\limits^b_a f'_x dt$.
                
                $\frac{\partial f}{\partial x_i} = \int\limits^1_0 v_i \left( A + t \left( x - A \right) \right) + \sum \frac{\partial v_i}{\partial x_j} \left( A + t \left( x - A \right) \right) t \left( x_i - A_i \right) dt = \int\limits^1_0 \left( t v_j \left( A + t \left( x - A \right) \right) \right)'_t dt - t v_j \left( A + t \left( x - A \right) \right) \bigg|^{t = 1}_{t = 0} = v_j(x)$.
        
        \subsection{Следствие к лемме Пуанкаре}
        
            $O$~--- открытое множество в $\RM$, $V \in C^1(O)$ и верное $\forall k$, $l : \frac{\partial v_k}{\partial x_l} = \frac{\partial v_l}{\partial x_k}$, тогда оно локально-потенциальное.
            
            \subsubsection{Доказательство}
            
                $I(v, \gamma) = \int\limits^b_a \langle V \left( \gamma(t) \right), \gamma'(t) \rangle dt$.
    \newpage
    
    \section{Лемма о гусенице}
    
        $O \subset \RM$, $\forall x \in O$ задана окрестность $U(x)$ и $\gamma : [a, b] \rightarrow O$~--- непрерывный путь. Тогда существует дробление $a = t_0 < t_1 < t_2 < \ldots < t_n = b$ и шары $B_k \subset O$, $\forall k \in [1, n] : \gamma \big|_{[t_{k - 1}, t_k]} \subset B_k$.
        
        \subsection{Доказательство}
        
            $\forall c \in [a, b]$ фиксируем $B_c = B \left( \gamma(c), r_c \right) \subset U \left( \gamma (c) \right)$.
            
            $\overline{\alpha}_c = \inf \left( \alpha \in [a, b] : \gamma \bigg|_{[a, c]} \subset B_c \right)$,
            
            $\overline{\beta}_c = \sup \left( \beta \in [a, b] : \gamma \bigg|_{[c, b]} \subset B_c \right)$.
            
            Заузим $\overline{\alpha}_c < \alpha_c < c < \beta_c < \overline{\beta}_c$, $\bigcup ( \alpha_c, \beta_c )$~--- открытое покрытие $[a, b]$.
            
            В точке $c = a$ $\alpha_c = a$, в точке $c = b$ $\beta_c = b$, $[a, b] \subset \bigcup\limits_{finite} \left( \alpha_c, \beta_c \right)$. 
            
            Удалим лишние наложение, т.е. удалим такие пары $(\alpha_i, \beta_i) \subset \bigcup\limits_{i \neq j} (\alpha_j, \beta_j)$. Тогда $\forall (\alpha_c, \beta_c)$ существует уникальная точка $d_c \in (\alpha_c, \beta_c$ и $\gamma \bigg|_{[t_{k - 1}, t_k]} \subset B_{C_k}$.
            
    \newpage
    
    \section{Лемма о равенстве интегралов по похожим путям}
    
        $V : O \rightarrow \RM$~--- локально потенциальное векторное поле, $\gamma$, $\overline{\gamma}$~--- похожие, кусочно гладкие пути, $\gamma(a) = \overline{\gamma}(a)$ и $\gamma(b) = \overline{\gamma}(b)$, тогда
        
        $\int\limits_{\gamma} \sum V_i dx_i = \int\limits_{\overline{\gamma}} \sum V_i dx_i$.
        
        \subsection{Доказательство}
        
            Берём $V$-гусеницу, $f_k$~--- потенциал в $B_k$, необходимо согласовать потенциалы, $f_k = f_{k + 1}$ на $B_k \cap B_{k + 1}$,
            
            $\int\limits_{\gamma} \sum v_i dx_i = \sum\limits^n_{k = 1} \int \left( v_1 dx_1 + \ldots + v_m dx_m \right) = \sum f_k \left( \gamma(t_k) \right) - f_k \left( \gamma (t_{k - 1}) \right) = f \left( \gamma(b) \right) - f \left( \gamma(a) \right)$.
            
    \newpage
    
    \section{Лемма о похожести путей, близких к данному}
    
        $\gamma : [a, b] \rightarrow O \subset \RM$. Тогда $\exists \delta > 0$, если $\overline{\gamma}$ и $\overline{\overline{\gamma}} : [a, b] \rightarrow 0$, таковы, что $\forall t \in [a, b]$ $|\gamma(t) - \overline{\gamma(t)}| < \delta$ и $|\gamma(t) - \overline{\overline{\gamma}}(t)| < \delta$,
        
        то $\gamma$, $\overline{\gamma}$ и $\overline{\overline{\gamma}}$~--- похожи.
        
        \subsection{Доказательство}
        
            Берём $V$-гусеницу для $\gamma$, тогда $\gamma \left( [t_{k - 1}, t_k ] \right)$~--- компактное множество в $B_k$, тогда $\exists \delta_k$~--- окрестность этого компакта в $B_k$, возьмём $\delta := \min\limits_{1 \leq k \leq n} \delta_k$.
            
    \newpage
    
    \section{Равенство интегралов по гомотопным путям}
    
        $V$~--- локально потенциальное векторное поле в $O \subset \RM$, $\gamma_0$ и $\gamma_1$~--- гомотопно связанные, тогда $I(V, \gamma_0) = I(V, \gamma_1)$.
        
        \subsection{Доказательство}
        
            $\Gamma$~--- гопотопия, $\gamma_u(t) = \Gamma(t, u)$, $t \in [a, b]$ и $u \in [0, 1]$. $\Phi(u) = I(V, \gamma_1)$. Проверим $\Phi$~--- локально постоянно, т.е. $\forall u_0 : \exists w(u) : \forall u \in w(u_0) \cap [0, 1]$ верно $\Phi(u) = \Phi(u_0)$. $\Gamma$~--- равномерное непрерывно, тогда
            
            $\forall \delta > 0 : \exists \sigma > 0 : \forall t, t' : | t - t' | < \sigma$ и $\forall u, u' : | u - u' | < \sigma$ выполнено $\left| \Gamma(t, u) - \Gamma(t', u') \right|  \frac{\delta}{2}$.
            
            Берём $\delta$ из предыдущей леммы для пути $\gamma_{u_0}$ и $(u - u_0) < t$ для любого $t \in [a, b]$, $\left| \Gamma(t, u) - \Gamma(t, u_0) \right| < \frac{\delta}{2}$ и $\left| \gamma_u(t) - \gamma_{u_0}(t) \right| < \frac{\delta}{2} \Rightarrow \gamma_u$ и $\gamma_{u_0}$~--- похожи. Подберём $\overline{\gamma}_u$ и $\overline{\gamma}_{u_0}$~--- кусочно гладкие, $\frac{\delta}{4}$ близко к $\gamma_u$ и $\gamma_{u_0}$, $\forall t : | \gamma(t) - \overline{\gamma}(t) | < \delta$, значит $\overline{\gamma}_u$ и $\overline{\gamma}_{u_0}$~--- похожи.
            
            $I(V, \gamma_u) = I(V, \overline{\gamma}_{u_0}) = I(V, \overline{\gamma}_{u_0}) = I(V, \gamma_0)$.
    
    \newpage
    
    \section{Теорема о  резиночке}
    
        $O = \mathbb{R}^2 \setminus \left\{ (0, 0) \right\}$, $\gamma : [0, 2 \pi] \rightarrow O$, $\gamma(t) \rightarrow (\cos{t}, \sin{t})$~--- петля. Тогда эта петля нестягиваема.
        
        \subsection{Доказательство}
        
            $V(x, y) := \left( \dfrac{-y}{x^2 + y^2}, \dfrac{x}{x^2 + y^2} \right)$, $\dfrac{\partial v_1}{x} = \dfrac{\partial v^2}{y} \rightarrow V$~--- локально потенциальное,
            
            $I(V, \gamma) = \int\limits^{2 \pi}_0 \dfrac{-\sin{t}}{\cos^2{t} + \sin^2{t}} (-\sin{t}) + V_1(\gamma(t))\gamma'_1(t) + \dfrac{\cos{t}}{1} \cos{t}dt = 2 \pi \neq 0$.
            
    \newpage
    
    \section{Теорема Пуанкаре для односвязной области}
    
        $O$~--- односвязная область в $\RM$, $V$~--- локально потенциальное векторное поле в $O$. Тогда $V$~--- потенциально в $O$.
        
        \subsection{Доказательство}
        
            $\gamma_0$~--- кусочно-гладкий замкнутый путь, значит гомотопен постоянному пути, значит $I(V, \gamma_0) = I(V,$ постоянному пути $) = 0$. т.е. выполняется критерий потенциальности вектроного поля.
            
    \newpage
    
    \section{Свойства объема: усиленная монотонность, конечная полуаддитивность}
    
        $\mu : \mathcal{P} \rightarrow \overline{\mathbb{R}}$. Тогда
        
        \begin{enumerate}
        
            \item Усиленная монотонность: $\forall A$, $A_1$, $A_2$, $\ldots$, $A_n \in \mathcal{P}$, все $A_i$~--- дизъюнкты и $\bigcup\limits^n_{i = 1}A_i \subset A$, тогда
            
                $\sum\limits^n_{i = 1} \mu A_i \leq \mu A$;
                
            \item Конечная полуаддитивность: $\forall A$, $A_1$, $A_2$, $\ldots$, $A_n \in \mathcal{P}$, все $A_i$~--- дизъюнкты и $A \subset \bigcup\limits^n_{i = 1} A_i$, тогда
            
                $\mu A \leq \sum\limits^n_{i = 1} \mu A_i$;
                
            \item $A$, $B$, $\left( A \setminus B \right) \in \mathcal{P}$, $\mu B < +\infty \Rightarrow \mu \left( A \setminus B \right) \geq \mu A - \mu B$.
            
        \end{enumerate}
        
        \subsection{Доказательство}
        
            \begin{enumerate}
            
                \item $A \setminus \bigsqcup\limits^n_{i = 1} A_i = \bigsqcup\limits_{finite} B_l$, $B_l \in \mathcal{P}$, $A = \bigsqcup\limits^n_{i = 1} A_i \sqcup \bigsqcup\limits_{finite} B_k$, $\mu A = \sum \mu A_i + \sum \mu B_l \geq \sum \mu A_i$.
                
                \item $B_k = A \cap A_k \in \mathcal{P}$, $A = \bigcup\limits^n_{k = 1} B_k$~--- сделаем это объединением дизъюнктов, $C_1 := B_1$, $C_2 := B_2 \setminus B_1$ и т.д., $A = \bigsqcup\limits_{k, j} D_{kj}$, $\mu A = \sum \mu D_{kj}$, фиксируем $k$ и получаем $\sum \mu D_{kj} = \mu C_k \leq \mu B_k \leq \mu A_k$ и получаем $\mu A \leq \sum \mu A_k$;
                
                \item
                    
                    \begin{itemize}
                    
                        \item $B \subset A$ и $\mu \left( A \setminus B \right) + \mu B = \mu A$;
                        
                        \item $B \not\subset A$, $A \setminus B = A \setminus (A \cap B)$, причём $A \cap B \in \mathcal{P}$, и
                        
                            $\mu \left( A \setminus B \right) = \mu A - \mu \left( A \cap B \right) \geq \mu A - \mu B$.
                            
                    \end{itemize}
                    
            \end{enumerate}
            
    \newpage
    
    \section{Теорема об эквивалентности счетной аддитивности и счетной полуаддитивности}
    
        $\mu : \mathcal{P} \rightarrow \overline{\mathbb{R}}$, объём. Тогда эквивалентны:
        
        \begin{enumerate}
        
            \item $\mu$~--- счётная аддитивная (т.е. $\mu$~--- мера);
            
            \item $\mu$~--- счётная полуаддитивность : $A$, $A_1$, $A_2$, $\ldots \in \mathcal{P}$ и $A \subset \bigcup\limits^{+\infty}_{i = 1} A_i \Rightarrow \mu A \leq \sum \mu A_i$.
            
        \end{enumerate}
        
        \subsection{Доказательство}
        
            \begin{itemize}
            
                \item $1 \Rightarrow 2$
                
                    $5$ формул $A = \bigcup A_k$ (будут написаны позже)
                    
                \item $2 \Rightarrow 1$
                
                    $A = \bigsqcup A_i$, надо проверить $\mu A = \sum \mu A_i$, усиленная монотонность: $\mu A \geq \sum\limits^n_{i = 1} \mu A_i$, по условию $\mu A \leq \sum\limits^{+\infty}_{i = 1} \mu A_i \Rightarrow \mu A = \sum \mu A_i$.
                        
            \end{itemize}
            
    \newpage
    
    \section{Теорема о непрерывности снизу}
    
        $\mathcal{A}$~--- алгебра, $\mu : \mathcal{A} \rightarrow \overline{\mathbb{R}}$~--- объём. Тогда эквивалентны следующие утверждения:
        
        \begin{enumerate}
        
            \item $\mu$~--- мера, т.е. выполняется счётная аддитивность;
            
            \item $\mu$~--- непрерывна снизу, т.е. $A$, $A_1$, $A_2$, $\ldots \in \mathcal{A}$, $A_1 \subset A_2 \subset A_3 \subset \ldots$, $A = \bigcup\limits^{+\infty}_{i = 1} A_i$ и $\mu A = \lim\limits_{i \rightarrow +\infty} \mu A_i$.
            
        \end{enumerate}
        
        \subsection{Доказательство}
        
            \begin{itemize}
            
                \item $1 \Rightarrow 2$
                    
                    $B_1 := A_1$, $\ldots$, $B_k = := A_k \setminus \bigcup\limits^{k - 1}_{i = 1} A_i$, тогда $B_i$~--- дизъюнкты, тогда $A_k = \bigsqcup\limits^k_{i = 1} B_i$ и $a = \bigsqcup\limits^{+\infty}_{i = 1} B_i$, $\mu A = \sum\limits^{+\infty}_{i = 1} \mu B_i = \lim\limits_{n \rightarrow +\infty} \sum\limits^n_{i = 1} \mu B_i$.
                    
                \item $2 \Rightarrow 1$
                
                    $C = \bigsqcup C_i$, верно ли, что $\mu C = \sum\limits^{+\infty}_{i = 1} \mu C_i$, $A_k = \bigsqcup\limits^k_{i = 1} C_i$, $A_1 \subset A_2 \subset A_3 \subset \ldots$ и $\bigcup A_i = A$,
                    
                    $\mu A = \lim\limits_{N \rightarrow +\infty} \mu A_N = \lim \sum\limits^N_{i = 1} \mu C_i = \sum\limits^{+\infty}_{i = 1} \mu C_i$.
                    
            \end{itemize}
            
    \newpage
    
    \section{Теоремы о непрерывности сверху}
    
        $\mathcal{A}$~--- алгебра, $\mu : \mathcal{A} \rightarrow \overline{\mathbb{R}}$~--- конечный объём ($\mu X < +\infty$), тогда эквивалентны следующие утверждения:
        
        \begin{enumerate}
        
            \item $\mu$~--- мера, т.е. выполняется счётная аддитивность;
            
            \item $\mu$~--- непрерывна сверху, т.е. $A$, $A_1$, $A_2$, $\ldots \in \mathcal{A}$, $A_1 \supset A_2 \supset \ldots$, $A = \bigcap\limits^{+\infty}_{i = 1} A_i$, $\mu A = \lim\limits_{N \rightarrow +\infty} \mu A_N$;
            
            \item $\mu$~--- непрерывна сверху на пустом множестве, т.е. $A_1$, $\ldots$, $A_n$, $\ldots \in \mathcal{A}$, $A_1 \supset A_2 \supset A_3 \supset \ldots$, $\bigcap\limits^{+\infty}_{i = 1} A_i = \varnothing$.
            
        \end{enumerate}
        
        \subsection{Доказательство}
        
            \begin{itemize}
            
                \item $1 \Rightarrow 2$
                
                    $B = A_1 \setminus A$, $B_k := A_1 \setminus A_k$, тогда $B_1 \subset B_2 \subset B_3 \subset \ldots$ и $\bigcup B_k = B$, $\mu B = \lim\limits_{k \rightarrow +\infty} \mu B_k$, $\mu A_1 - \mu A = \lim\limits_{k \rightarrow +\infty} \left( \mu A_1 - \mu A_k \right) \Rightarrow \mu A = \lim\limits_{k \rightarrow +\infty} \mu A_k$.
                    
                \item $2 \Rightarrow 3$
                
                    Очевидно.
                    
                \item $3 \Rightarrow 1$
                
                    $C = \bigsqcup C_i$, проверить, что $\mu C = \sum \mu C_i$, $A_1 \supset A_2 \supset A_3 \supset \ldots$, $A_k = \bigsqcup\limits^{+\infty}_{i = k + 1} C_i = C \setminus \left( \bigsqcup\limits^k_{i = 1} C_i \right) \Rightarrow A_k \in \mathcal{A}$.
                    
                    $\bigcap A_k = \varnothing \Rightarrow \mu A_k \rightarrow 0$, $C = \bigsqcup\limits^k_{i = 1} C_i \sqcup A_k$ и $\mu C = \sum\limits^k_{i = 1} \mu C_i + \mu A_k$, $k \rightarrow +\infty \Rightarrow \mu C = \sum \mu C_i$.
                    
            \end{itemize}
            
\end{document}
